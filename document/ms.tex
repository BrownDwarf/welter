%\documentclass[iop,revtex4]{emulateapj}% change onecolumn to iop for fancy, iop to twocolumn for manuscript
\documentclass[twocolumn]{emulateapj}% change onecolumn to iop for fancy, iop to onecolumn for manuscript
%\documentclass[12pt,preprint]{aastex}

%\usepackage{lineno}
%\usepackage{blindtext}
%\linenumbers

\let\pwiflocal=\iffalse \let\pwifjournal=\iffalse
%From: http://arxiv.org/format/1512.00483
%\input{setup}
\usepackage{enumerate}
\usepackage{amsmath,amssymb}
\usepackage{bm}
\usepackage{color}
\usepackage[utf8]{inputenc}

%% For anyone who downloaded my source file from arxiv:
%% I stole most of this setup.tex from a paper by Peter .K.G. Williams, but I made a bunch of edits to satisfy my own needs. You might check his paper out (http://arxiv.org/abs/1409.4411) for the original source file or contact him if you have any questions, since I don't really understand how some of these things work. 
%One cool thing it does is you can define an object, just that when someone clicks on the pdf it will link to simbad. I could never quite get this to work, probably because you have to get the text exactly right and my motivation for getting it to work was not super high. 


% basic packages
\usepackage{amsmath,amssymb}
\usepackage[breaklinks,colorlinks,urlcolor=blue,citecolor=blue,linkcolor=blue]{hyperref}
\usepackage{epsfig}    
\usepackage{graphicx}    
\usepackage{lineno}
\usepackage{natbib}
\usepackage{bigints}
\usepackage[outdir=./]{epstopdf}



% font stuff
\usepackage[T1]{fontenc}
\pwifjournal\else
  \usepackage{microtype}
\fi


% emulateapj has overly conservative figure widths, I think because some
% people's figures don't have good margins. Override.
\pwifjournal\else
  \makeatletter
  \renewcommand\plotone[1]{%
    \centering \leavevmode \setlength{\plot@width}{0.95\linewidth}
    \includegraphics[width={\eps@scaling\plot@width}]{#1}%
  }%
  \makeatother
\fi


\makeatletter

\newcommand\@simpfx{http://simbad.u-strasbg.fr/simbad/sim-id?Ident=}

\newcommand\MakeObj[4][\@empty]{% [shortname]{ident}{url-escaped}{formalname}
  \pwifjournal%
    \expandafter\newcommand\csname pkgwobj@c@#2\endcsname[1]{\protect\object[#4]{##1}}%
  \else%
    \expandafter\newcommand\csname pkgwobj@c@#2\endcsname[1]{\href{\@simpfx #3}{##1}}%
  \fi%
  \expandafter\newcommand\csname pkgwobj@f#2\endcsname{#4}%
  \ifx\@empty#1%
    \expandafter\newcommand\csname pkgwobj@s#2\endcsname{#4}%
  \else%
    \expandafter\newcommand\csname pkgwobj@s#2\endcsname{#1}%
  \fi}%

\newcommand\MakeTrunc[2]{% {ident}{truncname}
  \expandafter\newcommand\csname pkgwobj@t#1\endcsname{#2}}%

\newcommand{\obj}[1]{%
  \expandafter\ifx\csname pkgwobj@c@#1\endcsname\relax%
    \textbf{[unknown object!]}%
  \else%
    \csname pkgwobj@c@#1\endcsname{\csname pkgwobj@s#1\endcsname}%
  \fi}
\newcommand{\objf}[1]{%
  \expandafter\ifx\csname pkgwobj@c@#1\endcsname\relax%
    \textbf{[unknown object!]}%
  \else%
    \csname pkgwobj@c@#1\endcsname{\csname pkgwobj@f#1\endcsname}%
  \fi}
\newcommand{\objt}[1]{%
  \expandafter\ifx\csname pkgwobj@c@#1\endcsname\relax%
    \textbf{[unknown object!]}%
  \else%
    \csname pkgwobj@c@#1\endcsname{\csname pkgwobj@t#1\endcsname}%
  \fi}

\makeatother


% Evil magic to patch natbib to only highlight the year paper refs, not the
% authors too; as seen in ApJ. From
% http://tex.stackexchange.com/questions/23227/.

\pwifjournal\else
  \usepackage{etoolbox}
  \makeatletter
  \patchcmd{\NAT@citex}
    {\@citea\NAT@hyper@{%
       \NAT@nmfmt{\NAT@nm}%
       \hyper@natlinkbreak{\NAT@aysep\NAT@spacechar}{\@citeb\@extra@b@citeb}%
       \NAT@date}}
    {\@citea\NAT@nmfmt{\NAT@nm}%
     \NAT@aysep\NAT@spacechar\NAT@hyper@{\NAT@date}}{}{}
  \patchcmd{\NAT@citex}
    {\@citea\NAT@hyper@{%
       \NAT@nmfmt{\NAT@nm}%
       \hyper@natlinkbreak{\NAT@spacechar\NAT@@open\if*#1*\else#1\NAT@spacechar\fi}%
         {\@citeb\@extra@b@citeb}%
       \NAT@date}}
    {\@citea\NAT@nmfmt{\NAT@nm}%
     \NAT@spacechar\NAT@@open\if*#1*\else#1\NAT@spacechar\fi\NAT@hyper@{\NAT@date}}
    {}{}
  \makeatother
\fi

\newcommand{\prob}{{\rm prob}}
\newcommand{\qN}{\{q_i\}_{i=1}^N}
\newcommand{\qM}{\{q_{im}\}_{i=1,m=0}^{N,M}}
\newcommand{\yN}{\{y_i\}_{i=1}^N}

\newcommand{\kms}{ \textrm{km s}^{-1} }

\newcommand{\vM}{\mathsf{M}}
\newcommand{\vD}{\mathsf{D}}
\newcommand{\vR}{\mathsf{R}}
\newcommand{\vC}{\mathsf{C}}
\newcommand{\fM}{ \vec{{\bm M}}}
\newcommand{\fMi}{M_i}
\newcommand{\fD}{ \vec{{\bm D}}}
\newcommand{\fDi}{D_i}
\newcommand{\fR}{ {\bm R}}
\newcommand{\dd}{\,{\rm d}}
\newcommand{\trans}{\mathsf{T}}
\newcommand{\teff}{T_\textrm{eff}}
\newcommand{\logg}{\log g}
\newcommand{\Z}{[{\rm Fe}/{\rm H}]}
\newcommand{\A}{[\alpha/{\rm Fe}]}
\newcommand{\vsini}{v \sin i}
\newcommand{\matern}{Mat\'{e}rn}
\newcommand{\HK}{$\textrm{H}_2$O-K2}
\newcommand{\cc}[2]{c_{#2}^{(#1)}} 

\newcommand{\flam}{f_\lambda}
\newcommand{\vt}{ {\bm \theta}}
\newcommand{\vT}{ {\bm \Theta}}
\newcommand{\vp}{ {\bm \phi}}
\newcommand{\vP}{ {\bm \Phi}}
\newcommand{\cheb}{ \vp_{\mathsf{P}}}
\newcommand{\chebi}[1]{ \vp_{\textrm{Cheb}_{#1}}}
\newcommand{\Cheb}{ \vP_{\textrm{Cheb}}}
\newcommand{\Chebi}[1]{ \vP_{\textrm{Cheb}_{\ne #1}}} 
\newcommand{\cov}{ \vp_{\mathsf{C}}}
\newcommand{\covi}[1]{ \vp_{\textrm{cov}_{#1}}} 
\newcommand{\Cov}{ \vP_{\textrm{cov}}}
\newcommand{\Covi}[1]{ \vP_{\textrm{cov}_{\ne #1}}} 

\newcommand{\allParameters}{\vT} 
\newcommand{\nuisanceParameters}{\vP} 

\newcommand{\KK}{\mathcal{K}}
\newcommand{\Kglobal}{\KK^{\textrm{G}}}
\newcommand{\Klocal}{\KK^{\textrm{L}}}

\newcommand{\Gl}{Gl\,51}
\newcommand{\PHOENIX}{{\sc Phoenix}}

% Appendix commands
\newcommand{\wg}{\mathbf{w}^\textrm{grid}}
\newcommand{\wgh}{\hat{\mathbf{w}}^\textrm{grid}}

\newcommand{\Sg}{\mathbf{\Sigma}^\textrm{grid}}


\newcommand{\todo}[1]{ \textcolor{blue}{\\TODO: #1}}
\newcommand{\comm}[1]{ \textcolor{red}{SA: #1}}
\newcommand{\hili}[1]{ \textcolor{green}{#1}}
\newcommand{\ctext}[1]{ \textcolor{blue}{\% #1}}


%  From Peter Williams and Andy Mann again:
\newcommand{\um}{$\mu$m}


\newcommand{\iancze}{{\sc C15}}

\providecommand{\eprint}[1]{\href{http://arxiv.org/abs/#1}{#1}}
\providecommand{\adsurl}[1]{\href{#1}{ADS}}
\newcommand{\name}{LkCa 4 }
\newcommand{\project}[1]{\textsl{#1}}
%\def\vsini{$v\sin{i_*}$}


\slugcomment{In preparation}

\shorttitle{Starspots on LkCa 4}

\shortauthors{Gully-Santiago et al.}

\bibliographystyle{yahapj}

\begin{document}
 
\title{Placing the spotted T Tauri star LkCa 4 on an HR diagram}

\author{Michael A. Gully-Santiago,\altaffilmark{1} Gregory J. Herczeg,\altaffilmark{1} Ian Czekala,\altaffilmark{2} Garrett Somers,\altaffilmark{3} Konstantin Grankin,\altaffilmark{4} Kevin Covey,\altaffilmark{5} J.F. Donati,\altaffilmark{6} Sylvia Alencar,\altaffilmark{7} Gaitee Hussein,\altaffilmark{8} Ben Shappee,\altaffilmark{9} Greg Mace,\altaffilmark{10} Jae-Joon Lee\altaffilmark{11} T.~W.-S.~Holoien,\altaffilmark{12,13} Jessy Jose\altaffilmark{1} Jimmy Liu\altaffilmark{15}}


\altaffiltext{1}{Kavli Institute for Astronomy and Astrophysics, Beijing, China}
\altaffiltext{2}{Stanford University, Palo Alto, CA}
\altaffiltext{3}{Department of Physics and Astronomy, Vanderbilt University, 6301 Stevenson Center, Nashville, TN, 37235, USA}
\altaffiltext{4}{TBD--Grankin}
\altaffiltext{5}{TBD--Covey}
\altaffiltext{6}{TBD--Donati}
\altaffiltext{7}{TBD--Alencar}
\altaffiltext{8}{TBD--Hussein}
\altaffiltext{9}{TBD--Shappee}
\altaffiltext{10}{Department of Astronomy, The University of Texas at Austin, 2515 Speedway St, Austin, TX, USA}
\altaffiltext{11}{TBD--Lee}
\altaffiltext{12}{Department of Astronomy, The Ohio State University, 140 West 18th Avenue, Columbus, OH 43210, USA}
\altaffiltext{13}{Center for Cosmology and AstroParticle Physics (CCAPP), The Ohio State University, 191 W. Woodruff Ave., Columbus, OH 43210, USA}
\altaffiltext{15}{TBD--Liu}


\begin{abstract}
Ages and masses of young stars can be estimated by comparing observed luminosities and effective temperatures to pre-main sequence stellar evolutionary model tracks.  Nonstandard physics like magnetic fields and starspots confound interpretations of pre-main sequence HR diagrams and their age and mass estimates.  
We characterize the fundamental properties of the spotted weak-lined \emph{T-Tauri} star LkCa 4 by searching for signatures of the emergent radiation originating from the starspot or starspot groups.  The spectral and photometric properties of this radiation can be used to constrain both the starspot filling factor and the starspot temperature.
A generalization of line depth ratio analysis is developed and applied to a high-resolution near-IR IGRINS spectrum.  The IGRINS spectrum is forward modeled with pre-computed \PHOENIX\ synthetic spectra using the open source spectral inference framework \texttt{Starfish} extended to include a two component mixture model comprised of ambient photosphere and starspot emission.  
Spectral features attributable to both ambient photosphere $\teffa\sim4100$ K and cool starspots $\teffb\sim2700-3000$ K are detected in the IGRINS spectrum.  In some spectral regions, the starspot emission acts merely as a relatively featureless ``veiling'' continuum owing to the large rotational broadening ($\vsini\sim28$ km~s$^{-1}$) and heavy line-blanketing in cool star spectra.  The filling factor of of starspots at minimum light $\Delta V = 0.5$must be $>37\%$, and is likely much larger based on geometry and temperature considerations.  The IGRINS spectra yield consistently large ($>50\%$) fill factors across all IGRINS spectral orders, with a best fit filling factor of $f\sim80\%$ starspots at the epoch of spectrum acquisition.  The interpretation of the $\sim20\%$ ambient photosphere as a persistent ``hot spot'' is also possible.  The spectral energy distribution, color variability, TiO variability, rotational modulation, and low-resolution optical/IR spectroscopy show broad support for the starspot parameter estimates derived separately from IGRINS, strengthening the case for a large albeit secularly changing starspot filling factor and temperature contrast through time.
The revised effective temperature and luminosity make LkCa 4 appear much lower mass and much younger compared to previous estimates based on optical spectroscopy and not taking into account the large covering fraction of the cool spots.  Emission from longitudinally symmetric starspots has likely caused the misplacement of spotted young stars on the observational HR diagram, leading to incorrect estimates for ages and masses and contributing, in part, to the apparent age spread observed towards young clusters.
\end{abstract}

\keywords{stars: fundamental parameters --- stars: individual (LkCa 4) ---  stars: low-mass -- stars: statistics}

\maketitle

\section{Introduction}\label{sec:intro}

The causes of global age uncertainties and of large luminosity spreads in individual clusters are controversial.  Observationally, \citet{hartmann01} and \citet{slesnick08} argue that measurement uncertainties may mask any real differences in ages within a cluster.  In models, \citet{hartmann97} and \citet{baraffe09} describe how variable accretion histories change the stellar contraction and hence luminosities at early times.  Contraction rates also depend on the prescription for convection, which may vary with mass.

Magnetic activity is a likely source of significant uncertainty in both the models and observations of young low-mass stars. Convection at these young ages generates strong magnetic fields, as measured by Zeeman broadening and polarimetry \citep[e.g.][]{johnskrull07,donati09} and as seen in starspots \citep[e.g.][]{stauffer03,grankin08}.  Evolutionary models are just now starting to implement new prescriptions for convection with magnetic fields (\citet{somers15}, \citet{feiden16}; see also \citet{baraffe15} for an updated treatment of convection without introducing magnetic fields).  Stellar evolution models including the effect of starspots can make a coeval 10 Myr population exhibit apparent age spreads of 3$-$10 Myr, with standard mass estimates being biased towards lower masses \citep{somers15,feiden16}.  Starspots may also be responsible for biases in stellar effective temperatures derived by different methods.  For example, the effective temperatures for 3493 young stars measured using the APOGEE spectrograph \citep[$1.5-1.70 \;\mu$m at $R=22,500$][]{wilson10} exhibit systematic offsets from prior measurements predominantly that were predominantly made at optical wavelengths.  Typical offsets range from $-400-400$ K, with a systematic dependence on $\teff$, with some offsets as large as 1000 K \citep{cottaar14}.  

% Theory of starspots
% Jackson \& Jeffries 2014ab needs to go somewhere.


Starspots have recently been observed with exceptional photometric precision from monitoring surveys targeting exoplanet transits \citep[\emph{e.g}][]{harrison11,davenport15}.  The single band light curves from such surveys cannot separate the relative areas and temperatures of the emitting regions, and therefore cannot yield estimates for the filling factor of spots without making assumptions about the relative temperature contrast. Contemporaneous or near-contemporaneous panchromatic photometric monitoring \citep{petrov94,bouvier95,grankin07,cody14} can break the degeneracy between starspot filling factor and starspot temperature but these measurements are still limited to detecting large longitudinally asymmetric starspots.  Transiting planets or eclipsing binaries passing in front of starspots \citep{desert11} can break geometric degeneracies, providing estimates for the characteristic sizes and lifetimes of starspots or starspot groups, but the number of such transiting systems and demands of high photometric precision limit wider application.  Zeeman Doppler Imaging \citep[ZDI]{donati14} suffers from similar longitudinal asymmetry limitations as photometry, but can yield some information on the latitudinal distribution of starspots in reconstructed brightness maps based on models for spectro-polarimetric line shapes.  All of these techniques would tend to underestimate the starspot areal coverage of pole-on stars, stars with longitudinally symmetric bands of starspots, or stars with an isotropic distribution of small starspots.  

Spectroscopic strategies have the power to measure both the starspot areal coverage and temperature in any geometric configuration.  At low resolution, SED modeling \citep{wolk96} has detected evidence for sunspots in weak-lined T-Tauri stars (WTTS).  At high resolution, line depth ratio analysis has been demonstrated mostly on evolved stars in relatively small bandwidths, namely the TiO bands \citep{neff95,oneal96,oneal98,oneal04}, or the OH 1.563 $\mu$m feature \citep{oneal01}.  Past spectroscopic detections of spots on young stars have focused on TW Hya, an active accretor, and DQ Tau, a close binary in which both components are accreting \citep{debes13,bary14}.  

The WTTS LkCa 4 \citep{herbig86,strom89a,downes88,strom89b} is an ideal exemplar for a spotted pre-MS star because it does not have any veiling \citep[\emph{e.g.}][]{hartigan95} or mid-IR or mm excess \citep[\emph{e.g.}][]{andrews05,furlan06,buckle15}, and is not actively accreting \citep[\emph{e.g.}][]{edwards06,cauley12}.  LkCa 4 has no evidence for a close companion from both direct imaging searches\citep{karr10,kraus11,daemgen15}\footnote{The status for wide companions is less clear; see \citet{stauffer91,itoh08,kraus09,kraus11,herczeg14}} and RV searches \citep{nguyen12,donati14}.  This single star demonstrates a large amplitude of sinusoidal photometric variability indicative of large spots \citep{grankin08,xiao12}.  The variability amplitude cannot arise from eclipsing binarity, since the large RV variations would have been seen in spectroscopic monitoring.  All of these observations indicate that the spectrum of \name should be devoid of complicating factors like near-IR excess veiling, accretion excess, or close binaries.    

LkCa 4 has recently been examined with ZDI \citep{donati14}, revealing a complex distribution of cool and warm spots in brightness map reconstructions.  Dark polar spots extend to about $30^\circ$ from the pole, with about 5 appendages reaching down to about $60^\circ$ from the pole.  There is also evidence for a hot spot in the reconstructed ZDI map.  Still, unanswered questions about the large disagreement in LkCa 4's effective temperature remain, with 4100 K measured from holistic analysis of ESPaDOnS spectra yielding $\teff=4100$K \citep{donati14}, and analysis of the TiO bands of the same spectra yielding $\teff=3600$ K \citep{herczeg14}.

In this work, we measure the starspot properties of LkCa 4 with four complementary techniques.  Section \ref{sec:obs} quantifies the optical variability of LkCa 4 over the last 31 years based on all available photometric monitoring; a collection of spectral observations at various phases of variability is introduced.  Section \ref{sec:methods} and Appendix \ref{methods-details} describe a spectral inference technique aimed at generalizing line depth ratio analysis, which is applied to a high resolution $H$ and $K$ near-IR spectrum of LkCa 4 in Section \ref{sec:two_tempIGRINS}.  Section \ref{sec:GJHsection4} shows results from SED fitting, polychromatic photometric monitoring, and optical TiO line-depth ratio fitting.  All lines of evidence are ultimately combined to build a consistent picture of the spectral and temporal evolution of LkCa 4, and what can be understood about pre-MS stellar evolution from this exemplar.


\section{Observations}\label{sec:obs} 

\subsection{IGRINS Spectroscopy}\label{sec:igrins} 
We acquired observations of LkCa 4 with the Immersion Grating Infrared Spectrograph, IGRINS \citep{park14} on the Harlan J. Smith Telescope at McDonald Observatory on 2015-11-18 $09^h$ UTC.  IGRINS is a high resolution near-infrared echelle spectrograph providing simultaneous $R\simeq45,000$ spectra over 1.48-2.48 \um.  The spectrograph has two arms with 28 orders in $H-$band and 25 orders in $K-$band.  The data were reduced with the Pipeline Package \citep{jaejoonlee15}.  Telluric correction was performed by dividing the spectrum of \name by a spectrum of HR 1237, an A0V star observed immediately before \name, both at airmass 1.1.  The broad hydrogen lines in the A0V star produced broad flux excess residuals in the spectrum of LkCa 4.  No effort was made to mitigate these flux excess residuals, which affect several spectral orders.  These orders were ultimately excluded from further analyses.


\subsection{ESPaDOnS Spectroscopy}
We used ESPaDOnS on CFHT to obtain twelve high resolution optical spectra of \name from 8-21 Jan.~2014 as part of the MaTYSSE Large Program.  These spectra cover 3900$-$10000 \AA\ at $R\sim68,000$ and were obtained in spectropolarimetry mode.  The Zeeman Doppler Imaging obtained from these observations were analyzed by \citet{donati14}.  In this paper, we concentrate on the intensity spectrum.  To calculate TiO indices, the relative flux calibration was obtained from an approximate instrument sensitivity calculated from observations of 72 Tau.  The spectral shapes match the flux calibrated low resolution spectra described in \S 2.3.


\subsection{Low-resolution optical and near-IR spectra}

A flux-calibrated low-resolution optical spectrum of LkCa 4 was obtained using Palomar/DBSP \citep{oke82} on 30 Dec. 2008.  These spectra cover 3200--8700 \AA\ at a spectral resolution $R\sim 500$.  The data reduction and a spectral analysis are described by \citet{herczeg14} within the context of a larger survey.

Near-infrared spectra of LkCa 4 were obtained at ~01:30 UT on Dec. 30, 2008, using the TripleSpec spectrograph on the Astrophysical Research Consortium (ARC) 3.5 meter telescope at Apache Point Observatory in Sunspot, New Mexico. LkCa 4 was observed for a total integration time of 90s using an ABBA dither sequence along the slit to remove sky emission by differencing sequential exposures.  With the 1.1$\arcsec$ slit used for these observations, TripleSpec provides nearly contiguous coverage from 0.95 to 2.46 $\mu$m at a resolution of  R $\sim$3200 \citep{wilson04}. Spectra were reduced with a variant of the SpeXTool pipeline, originally developed by \citet{cushing04} and modified for use with TripleSpec data. Spectra were differenced, flattened, extracted, and wavelength calibrated prior to telluric correction and flux calibration; the latter two corrections were performed using the XTELLCOR IDL package \citep{vacca03} and a spectrum of HD 25175, a nearby A0V star observed directly after LkCa 4 at a similar airmass. 


\subsection{Photometric monitoring}

An archive of 1216 $B$, $V$ and/or $R$ photometric monitoring visits was assembled from published and unpublished data sources spanning 31 years in 22 observing seasons.  Published data targeting \name is composed of 29 $BVR$ visits from 1985$-$1986 \citep{vrba93}, 26 $BVR$ epochs from 1990$-$1991 \citep{bouvier93}, 284 visits in $V$-band (278 with $B$ and 268 with $R$) from 1992 August - 2004 October \citep{grankin08}, and 10 $BVR$ visits from 2013 \citep{donati14}.  ASAS3 \citep{pojmanski04} acquired 63 $V-$band measurements from 2002$-$2004.  Integral-OMC obtained 138 $V-$band measurements from 2006$-$2008 \citep{garzon12}.  The AAVSO archive \citep{kafka16} includes 385 $V$, 23 $B$ and 10 $R$ measurements from 2013$-$2016.  Unpublished data from the ASAS-SN survey \citep{shappee14} were obtained from 2012 January -- 2016 March in 186 visits.  Finally, recent photometry from the ongoing Crimean Astrophysical Observatory (CrAO) ROTOR project \citep{grankin08} is presented here for the first time.  The CrAO data includes 43 $BVR$ visits from August 2015 - March 2016. Figure \ref{fig:PhotTime} displays all the $V-$band photometry over the interval 1985-2016.  HJD, BJD, and JD are simply referred to as JD, which is accurate to the precision of available data.


\begin{figure*}
 \centering
 \includegraphics[width=0.98\textwidth]{figures/LkCa4_phot1986-2016.pdf}
 \caption{Overview of \name $V-$band photometric monitoring from 1986$-$2016.  The vertical lines denote the observing epochs of 2MASS, IGRINS, ESPaDOnS, DoubleSpec, and TripleSpec.  The near contemporaneous DoubleSpec and TripleSpec epochs lay on top of each other on this scale, as do the 12 ESPaDOnS epochs.  The abscissa range is equal to the current lifespan of the first author of this paper.}
 \label{fig:PhotTime}
\end{figure*}

Figure \ref{fig:PhotPhase} shows all available $V-$band data grouped by the 22 observing seasons and phase-folded by the period $P=3.375$ days obtained from multiterm Lomb-Scargle periodograms \citep{ivezic14}.  The general appearance of the phase-folded lightcurves does not change with perturbations to the period on the scale of 0.003 days.  The estimated $BVR$ magnitudes at the time of the spectral observations are determined from regularized multiterm fits \citep[\emph{i.e.} Fourier series truncated to the first $\sim 4$ components]{vanderplas15a} shown as the solid blue lines in Figure \ref{fig:PhotPhase}.  Table \ref{tbl_estimated_V} lists the estimated $BVR$ photometry and the observing epoch for data from IGRINS, ESPaDOnS, 2MASS \citep{skrutskie06}, DoubleSpec, and TripleSpec.  

\begin{figure*}
 \centering
 \includegraphics[width=0.95\textwidth]{figures/Vband_22s.pdf}
 \caption{Phase-folded lightcurves constructed assuming $P=3.375$ days for all 22 observing seasons.  The blue solid lines show a ``multiterm'' regularized periodic fit, that is, keeping the first $M_{\rm max}=4$ Fourier components \citep{vanderplas15a}.  The vertical lines show the epochs at which spectra or ancillary photometry were obtained, with the same line styles and colors as Figure \ref{fig:PhotTime}.  LkCa 4 shows secular changess in its light curve morphology.  The IGRINS spectrum was acquired near the median flux level, not the extrema.}
 \label{fig:PhotPhase}
\end{figure*}


\begin{deluxetable}{rrl}

\tabcolsep=0.11cm
\tablecaption{Estimated $V-$band magnitudes\label{tbl_estimated_V}}
\tablewidth{0pt}
\tablehead{
\colhead{JD $-$ 2456000} &
\colhead{$\hat V$} &
\colhead{Instrument}
}
\startdata
       665.7204 &  12.83 &   ESPaDoNs \\
       666.8505 &  12.68 &   ESPaDoNs \\
       667.7727 &  12.82 &   ESPaDoNs \\
       668.8699 &  12.90 &   ESPaDoNs \\
       672.8995 &  12.76 &   ESPaDoNs \\
       673.8408 &  12.66 &   ESPaDoNs \\
       674.7746 &  12.95 &   ESPaDoNs \\
       675.7396 &  12.86 &   ESPaDoNs \\
       676.7954 &  12.70 &   ESPaDoNs \\
       677.8699 &  12.81 &   ESPaDoNs \\
       678.7419 &  12.98 &   ESPaDoNs \\
       678.8950 &  12.93 &   ESPaDoNs \\
       990.7904 &  12.72 &     IGRINS \\
      1344.8610 &  12.83 &     IGRINS \\
\enddata

\tablecomments{See ref blah for more info.}
\end{deluxetable}


\section{FITS TO HIGH RESOLUTION SPECTRA}\label{sec:Starfish}

Starspots possess a spectrum distinct from the ambient photosphere.  Approaches like ZDI and line-bisector analysis \citep[\emph{e.g.}][]{prato08, donati14} target the modulation of stellar photospheric lines as starspots enter and exit the observable stellar disk.  In contrast, the technique described in this Section detects and measures the emergent spectrum originating from cool starspots.

Starspot line depth ratio analysis has traditionally been limited to isolated portions of spectrum that possess spectral lines attributable only to starspots and spectral lines attributable only to ambient photosphere \citep[\emph{e.g.}][]{neff95, oneal01}.  The apparent veiling of the lines and their respective temperature dependences can be combined to solve for the starspot and ambient photosphere temperatures and relative areal coverages.  This line depth ratio method suffers from the need to identify portions of the spectrum that possess such strong lines, and from the need to assemble large atlases of observed spectral templates to which the spotted star spectrum can be compared.  In this section, we introduce a generalization of the line depth ratio analysis which employs pixel-by-pixel modeling of all spectral orders.  By using all the spectral data, this strategy has the power to constrain starspot properties at relatively low filling factors and to identify weak lines that originate from the starspots.

We took two approaches to characterizing the effective temperature response from a spotted-star spectrum.  

Using the spectral fitting approach introduced in Section 3.1 \ref{sec:methods}, we first first separately fit the optical (ESPaDOnS) and near-IR (IGRINS) spectra with distinct, single $\teff$ models; these fits are described in Sections \ref{sec:ESP_starfish} and \ref{sec:IGR_starfish} respectively.  We then use a two temperature mixture model to perform a fit to the near-IR spectra in Section \ref{sec:two_tempIGRINS}.


First we measured a single unique effective temperature to each spectral order in the optical (\S \ref{sec:ESP_starfish}) and near-IR (\S \ref{sec:IGR_starfish}).  Second we fit a two-temperature mixture model (\S \ref{sec:methods}) to a near-IR echelle spectrum (\S \ref{sec:two_tempIGRINS}).  The near-IR was preferred over the optical, since the cool starspots emit most of their flux in the near-IR, enhancing the likelihood of direct detection of emission from the starspots.

\subsection{Methodology}\label{sec:methods} 

\citet[hereafter \iancze]{czekala15} developed a modular framework\footnote{The open source codebase and its full revision history is available at \url{https://github.com/iancze/Starfish}.  The experimental fork discussed in this paper is at \url{https://github.com/gully/Starfish}} to infer stellar properties from high resolution spectra.  The \iancze\ technique forward models observed spectra with synthetic spectra from pre-computed model grids.  The intra-grid-point spectra are ``emulated'' in a process similar but superior to interpolation, since it seemlessly quantifies the uncertainty attributable to the coarsely sampled stellar intrinsic parameters (see Appendix of \iancze).  The forward model includes calibration parameters, line spread functions, and a Gaussian process noise model to account for correlations in the residual spectrum.  We employed the \PHOENIX\ grid of pre-computed synthetic stellar spectra, which span a wide range of wavelengths at high spectral resolution with a sampling of 100 K in $\teff$ in our range of interest \citep{husser13}.  The modular framework was altered to accommodate starspot measurements in two ways. First, the single photospheric component was updated to include a starspot spectrum. Second, the MCMC sampling strategy was altered to accommodate the additional free parameters added by the starspot model.

The stellar photosphere is characterized as two photospheric components with a starspot temperature $\teffb$ and ambient photosphere temperature $\teffa$, with scalar solid angular coverages $\Omega_{\mathrm{spot}}$ and $\Omega_{\mathrm{amb}}$, respectively\footnote{For flux calibrated spectra, the total solid angle $\Omega$ can be constrained, but for typical echelle spectrographs only relative values of $\Omega_{\mathrm{spot}}$ and $\Omega_{\mathrm{amb}}$ can be inferred.}.  Starspots (or spot groups) are assumed to be cooler than the ambient photosphere, but otherwise share the same average intrinsic and extrinsic stellar parameters $\vsini, \logg, \Z, v_z$.  The composite mixture model for observed flux density is:
\begin{eqnarray} \label{eqn:mix_M}
S_{\mathrm{mix}} = \Omega_{\mathrm{amb}} B(\teffa)  + \Omega_{\mathrm{spot}} B(\teffb)
\end{eqnarray}
where $B(T)$ is the spectral radiance from model spectra.  The filling factor of the starspots is:
\begin{eqnarray} \label{eqn:fill_factor}
f_{\Omega} = \frac{\Omega_{\mathrm{spot}}}{\Omega_{\mathrm{amb}} + \Omega_{\mathrm{spot}}}
\end{eqnarray}

The term $f_{\Omega}$ represents an instantaneous, observational fill factor seen on one projected hemisphere, not $f_{\rm spot}$, the ``ratio of the spotted surface to the total surface areas'' \citep{somers15}.  In the limit of homogenously distributed spots, $f_{\Omega}$ tends to $f_{\rm spot}$, but in general $f_{\rm spot}$ is not a direct observable since some circumpolar regions of inclined stars will always face away from Earth; $f_{\Omega}$ can vary cyclically through rotational modulation, whereas $f_{\rm spot}$ changes secularly through starspot evolution that is not yet understood.

The starspot model includes two more free parameters than the standard \iancze\ model, namely $\teffb$ and $f_{\Omega}$.  The addition of these two parameters makes the MCMC sampling much more correlated than it was before because the relative contribution of the starspot and ambient photosphere are nearly degenerate over small changes in $\teffa$, $\teffb$, and $f_{\Omega}$.  This challenge motivated the second important change to the spectral inference framework, which involves technical aspects of switching from sampling the nuisance and stellar parameters separately in a blocked Gibbs framework to ensemble sampling using \texttt{emcee} \citep{foreman13}.  The affine-invariant \texttt{emcee} ensemble sampler is more resilient to correlations among stellar temperatures and fill factor than the Metropolis-Hastings sampler used in Gibbs sampling \citep{}.  The practical effect of this switch is that all 14 stellar and nuisance parameters are fit simultaneously in a single spectral order, making stellar parameter estimates $\vt_{o}$ unique for each spectral order $o$, whereas the \iancze\ strategy had the power to provide a single set of stellar parameters $\vt$ that was based on all $N_{\rm ord}$ spectral orders.  The $N_{\rm ord}$ sets of inferences on $\vt_{o}$ are combined by weighted averaging point estimates of results from reliable orders, which offers resilience to flagrant calibration artifacts or conspicuous model mismatches.  This process can be thought of as a coarse approximation to the much more sophisticated spectral line outlier rejection described in \iancze.  No attempt was made to downweight spectral-line outliers using local kernels.

This two-temperature mixture model for starspots requires single stars, or at least double stars with extremely large (>100) luminosity ratios or wide separations.  Unresolved single-lined or double-lined binaries could mimic a signal that would be misattributed to starspots.  As pointed out earlier, LkCa 4 has no detected companion from direct imaging, and its RV variations are consistent with starspot modulation, so any binary companion would have to have exceptionally large luminosity or mass ratios, rendering it undetectable with our spectral inference methodology with the finite signal to noise available in the IGRINS data.  LkCa 4's absense of mid-IR excess emission attributable to a disk also means that the spectral modeling requires no further parameters like veiling or accretion.

Further details of the spectral inference methodology are described in Appendix \ref{methods-details}.


\subsection{Single temperature fitting to the ESPaDOnS spectrum}\label{sec:ESP_starfish}

We performed spectral fitting on an ESPaDOnS spectrum acquired on 2014 January 11.  The spectrum was split into subsets of $N=35$ chunks, corresponding approximately to spectral order boundaries.  Fitting was performed separately on each of 35 spectral orders following the Metropolis-Hasting MCMC sampling procedure described in \iancze.  The spectral emulator was trained on stellar parameters in the range $\logg \in [3.5, 4.0]$, $\Z \in [-0.5, 0.5]$, and $\teff \in [3500, 4200]$.

The Metropolis-Hastings step sizes were tuned with several iterations of burn-in procedure and the final chains were visually checked for convergence, dropping 9 orders that failed to converge for numerical reasons.  We computed the median value and 5$^{th}$ and $95^{th}$ percentiles of burned-in subsets of the MCMC samples described above.  Overall the 26 sets of point estimates for stellar parameters show relatively good agreement, with exceptions.  The best-fitting spectral orders show effective temperature point estimates in the range $\teff=4000\pm130$ K.  Figure \ref{fig:SingleTeffvsOrder} displays the $\teff$ point estimates with 5$^{th}$ and $95^{th}$ percentile error bars placed at the central wavelength of each spectral order.

\begin{figure}
 \centering
 \includegraphics[width=0.48\textwidth]{figures/single_Teff_v_order}
 \caption{Effective temperature as derived from unique full spectrum fitting to each of 58 spectral orders in the optical through infrared portions of the spectrum and assuming a single component photosphere.  The effective temperature derived in the $K-$band is about 800 K lower than that derived in optical.  The increasing flux density from starspots compared to the flux from ambient photosphere can explain the observed trend in derived $\teff$.  The cluster of $H-$band orders at 1.7 $\mu$m correspond to a local peak in flux density of cool starspots.}
 \label{fig:SingleTeffvsOrder}
\end{figure}

\subsection{Single temperature fitting to the IGRINS spectrum}\label{sec:IGR_starfish}

We selected a subset of 32 of the 54 available IGRINS spectral orders\footnote{The finite computational cost limited running all the spectral orders.} with the low telluric absorption artifacts.  We performed full-spectrum fitting, again fitting unique stellar parameters $\vt = (\teff, \logg, \Z)$ for each spectral order $o$.  We used the same analysis procedure as described for the ESPaDOnS spectra, with one exception.  For the IGRINS $K-$band, we employed an expanded search range for the temperature, $\teff \in [3000, 4200]$ K, since the IGRINS $H-$ band demonstrated some saturation at the $\teff=3500$ K lower bound.  

We find a larger dispersion in the point estimates for the stellar parameters derived from the IGRINS data than those derived in optical.  The most conspicuous trend is in the derived effective temperature as a function of wavelength shown in Figure \ref{fig:SingleTeffvsOrder}.  The effective temperature peaks at values of $\sim4200$K in the short wavelength end of $H-$band and saturates at $<3500$ K at the long wavelength end of $H-$band.  The $K-$band shows even lower derived effective temperatures of $\sim3300$ K, or 700 K cooler than estimated from optical.  No single temperature can describe all the spectral lines present in the high resolution optical and IR spectra.  Sources with such discrepancies have been seen previously, for example in Figures 4 and 5 in \citet{bouvier92}.  These discrepancies are circumstantial evidence for the detection of spectral features attributable to starspots.


\subsection{Heightened sensitivity to starspot spectral lines in the infrared}\label{sec:whyNearIR}

Some care should be taken when directly comparing results between the ESPaDOnS and IGRINS spectra since they were not taken at the same time.  The particular ESPaDOnS spectrum used in Section \ref{sec:ESP_starfish} has an estimated $V-$band magnitude of 12.90 while the IGRINS spectrum has 12.83.  The fainter magnitude during the ESPaDOnS spectrum acquisition implies a greater coverage fraction than during the IGRINS spectrum.  The $\teff$ derived in the optical bands and short-wavelength end of $H$-band yield similar values of $\sim 4100$ K, suggesting the \emph{bulk}\footnote{The bulk appearance and disappearance of spectral lines is mostly controlled by the temperature of the emitting region of the local photosphere.  In other words, most of the variance in a spectrum is attributable to temperature, assuming near-solar metallicity.  Starspots imbue dearths of flux in the line profiles of optical spectra, but these are secondary to the mere presence of temperature-sensitive lines, despite carrying useful information about the longitudinal distribution of the spots.} spectral features are broadly consistent with a emission from a single temperature component. A single ESPaDOnS order surrounding the TiO bands shows an exceptionally low estimated $\teff$.  The long wavelength portion of $H-$band and all of $K-$band are more sensitive to starspot spectral signatures than the shorter wavelength portions.

The heightened sensitivity to starspot spectral lines as a function of wavelength can be understood in the following way.  Starspots are cooler than their surrounding photosphere and will, therefore, have a longer wavelength of peak emission.  The average ratio of flux density between starspot and bulk photosphere will increase with wavelength until asymptoting to a fixed value in the Rayleigh-Jeans tail \citep{wolk96}.  The bottom panel of Figure \ref{fig:SingleTeffvsOrder} shows the flux density ratio for patches of photosphere with equal areas (50\% filling factor) but different temperatures-- 2800 K and 4100 K for the starspot and ambient photosphere respectively.  The black-body ratios predict smooth flux ratio increases with wavelength, while the ratio of \PHOENIX\ models shows wavelength regions with heightened sensitivity to starspot fluxes, for example in $J-$band.

In summary, the visible portion of the spectrum will be dominated by the warm patches.  The infrared spectrum will have a lower overall spot contrast, and will have more net starspot flux than optical.


\subsection{Two-temperature fitting to IGRINS spectra}\label{sec:two_tempIGRINS}

Full-spectrum fitting was performed for 48 of the 54 available IGRINS spectral orders, omitting only the orders with the most pathological telluric spectral artifacts. We applied the mixture model as described in Section \ref{sec:methods} and Appendix \ref{methods-details}.  The MCMC samples generally had 5000 steps with 40 walkers.  The stellar parameter ranges were $\logg \in [3.5, 4.0]$ and $\Z \in [-0.5, 0.5]$; the \PHOENIX\ model spectrum temperature range\footnote{The \PHOENIX\ model spectra are indexed by $\teff$, but this term carries a different meaning for spotted stars than non-spotted stars.} was $\teffa, \teffb \in [2700, 4500]$.  $H-$band fits had normal distribution priors in place: solar metallicity to $\pm0.05$ dex, $logg=3.8\pm0.1$, and $\vsini=29\pm5$ km/s.  The MCMC chains all appeared to converge after about 1500 steps.

Marginalized distributions are computed for all 14 stellar and nuisance parameters by selecting the final 200 $\times$ 40 walkers = 8000 samples, and point estimates were obtained by computing the median, $5^{th}$ and $95^{th}$ percentiles of the marginal distributions.  The fit quality is first assessed by examining the consistency of the point estimates of $v_z$ and $\vsini$ across the spectral orders.  The distribution of $v_z$ and $\vsini$ exposed extremely poor fits in two orders ($o=91$ and $94$), with all other orders demonstrating $v_z = 12.4 \pm 2.6$ km/s and $\vsini = 28.8 \pm 2.0$ km/s.  

Overplotting forward-modeled spectra with the observed IGRINS spectra yielded insights on why some spectral orders performed better than others in assessing stellar properties.  This model comparison is performed by sorting the 8000 samples by their fill factor estimates and examining random draws of the cool spectrum, hot spectrum, and composite spectrum, excluding the Gaussian process coveriance and white noise steps.  Fitting defects were conspicuous.  Orders with extremely poor telluric correction residuals, large spectral line outliers, and uncorrected H line residuals from A0V standard division, are all excluded from our final stellar parameter compilation.  Notably, several orders in $K-$band were rejected due to large biases in metallicity, since $K-$band had uninformative priors on $\Z$, resulting in overfitting of orders possessing deep metal lines.  For example, orders $o=$ 79 and 81 overfit Ca~I and Na~I lines and orders $o=74-76$ overfit the CO lines, which are known for their gravity sensitivity \citep{rayner09}.  These and other metal lines can also be biased by Zeeman broadening, which could explain the heightened effect size with wavelength \citep{johnskrull07,deen13}.  Additionally, some orders without these conspicuous faults, but simply possessing mediocre fits, or relatively uninformative fits, were also removed from downstream estimates.  The remaining orders are rerelatively devoid of spectral line outliers and include the most information rich portion of the spectrum.  Figure \ref{fig:TwoTempResults} shows the distribution of $\teffa$ (blue shading), $\teffb$ (red shading), and $f_{\Omega}$ (yellow shading) for all of the spectral orders, with the rejected orders grayed out, and the reliable order subset shown in bold.  The point estimates for each spectral order are listed in Appendix Table \ref{tbl_order_results}.

\begin{figure*}
 \centering
 \includegraphics[width=0.95\textwidth]{figures/violin_Teff_fill_order} 
 \caption{Marginal probability distributions mirrored through the vertical axis \citep[``Violin plot'']{waskom14} for 48 IGRINS orders for $\teffa$ (blue shade), $\teffb$ (red shade), and fill factor $f_{\Omega}$ (yellow shade).  The stellar parameters are derived independently in each spectral order.  Spectral orders show differing levels of constraint on the starspot and ambient photosphere properties, with some ($o=104, 102, 100, 88, 83$) showing tight constraints on the filling factor of starspots.  The starspot temperature is consistent with values even lower than 2700 K, the lower limit of the temperature range used.  $K-$band orders show lower estimates for the ambient photosphere, though many of these estimates are unreliable due to spectral line outliers and uncorrected telluric residuals (light shaded distributions).}
 \label{fig:TwoTempResults}
\end{figure*}

The IGRINS spectrum demonstrates some features that are present only in the ambient photosphere model, and some features that are present only in the starspot model.  Figure \ref{fig:specPostageStamp} shows a selection of six such spectral features for a range of plausible fill factors.  In some cases (\emph{e.g.} the top two panels), featureless starspot spectra veil isolated spectral lines predicted in the ambient photosphere models.  In other cases (\emph{e.g.} the middle row and lower left panel), the ambient photosphere model veils sequences of shallow spectral features predicted in the starspot model.  The starspot spectral models predict shallow features because line blanketing from multiple indistinct molecular bands overlap from rotational broadening.  Any feature of interest will be biased to non-zero veiling.  

The combination of line blanketing and high $\vsini$ has probably hampered efforts to identify isolated spectral features suitable for line-depth-ratio analysis in the spectra of rotationally broadened young stars.  One such feature has been previously identified in the near-IR: the OH 1.563 $\mu$m line noted by \citet{oneal01} shows a clear pattern of 3 lines in our data, with the central line exceeding the depth of the adjacent two lines, although the \PHOENIX\ models predict a non-negligible ambient photosphere contribution to the middle line for our range of ambient and spot temperatures.  In comparison, the forward-modeling technique described in this work thrives in the presence of long sequences of indistinct yet predictably correlated spectral features.  The level of veiling of ambient photospheric lines is set by the starspot filling factor and temperature, while the level of veiling of the starspot lines is set by the filling factor and temperature of the ambient photosphere lines. Similar forward-modelling strategies have successfully identified patterns of weak metal lines in the line-blanketed spectra of low metallicity stars \citep{kirby11,kirby15}.  Figures \ref{fig:Hband3x7} and \ref{fig:Kband3x7} in the Appendix show 42 of the $H-$ and $K-$band spectra on a log scale with a single random composite model spectrum overplotted.  

\begin{figure*}
 \centering
 \includegraphics[width=0.45\textwidth]{figures/spectral_postage_stamp_04} 
 \includegraphics[width=0.45\textwidth]{figures/spectral_postage_stamp_01} 
 \includegraphics[width=0.45\textwidth]{figures/spectral_postage_stamp_05} 
 \includegraphics[width=0.45\textwidth]{figures/spectral_postage_stamp_06} 
 \includegraphics[width=0.45\textwidth]{figures/spectral_postage_stamp_02} 
 \includegraphics[width=0.45\textwidth]{figures/spectral_postage_stamp_03} 
 \caption{Examples of spectral features in the observed IGRINS spectrum (light gray line).  The composite spectrum model (purple thin line) is consistent with the observed spectrum for a range of fill factors, with examples of the median fill factor (middle panel of triptych) and $\pm2\sigma$ fill factors demarcated on the spectral postage stamps.  The upper right triptych shows a Zeeman-sensitive Mg I line that is modeled with no special attention to magnetic field, but is still coarsely reproduced in its gross appearance as a function of temperature, but potentially biasing estimates of $\teffa$ and/or $f_{\Omega}$ for individual spectral orders.}
 \label{fig:specPostageStamp}
\end{figure*}

Best fit values for the stellar parameters summarized are listed in Table \ref{tbl_adopted_props}.  Remarkably, the filling factor of starspots exceeds 50\%, with a best fit value of $f_{\Omega}=80\pm 5 \% $.

%%%%%%%%%%%%%%%%%%%%%%%%%%%%%%%%%%%%%%%%
% TABLE - History of LkCa4
%%%%%%%%%%%%%%%%%%%%%%%%%%%%%%%%%%%%%%%%
\begin{deluxetable}{cccc}

\tabcolsep=0.11cm
%\rotate
\tabletypesize{\footnotesize}
\tablecaption{Adopted values of \name from IGRINS spectra \label{tbl_adopted_props}}
\tablewidth{0pt}
\tablehead{
\colhead{$\teffa$} &
\colhead{$\teffb$} &
\colhead{$f_{\Omega}$} &
\colhead{$\vsini$} \\
\colhead{K} &
\colhead{K} &
\colhead{\%} &
\colhead{km s$^{-1}$}
}
\startdata
   $4200\pm100$ &     $2700-3000$ &    $75\pm10$ &   $28.5 \pm 1$ \\
\enddata
\tablecomments{These are the values at the epoch of the IGRINS spectrum acquisition.}
%\tablerefs{}

%\end{deluxetable*}
\end{deluxetable}


\section{The Two Temperature Fit to the SED and Stellar Rotation}\label{sec:GJHsection4}

\begin{figure}
 \centering
 \includegraphics[trim=3.1cm 13cm 1.0cm 3.6cm, clip=true, width=0.49\textwidth]{figures/phot_2750}
\caption{The SED of LkCa 4 (red) compared with the two-temperature spectrum obtained from the best fit to the IGRINS spectra (black).}
\label{fig:sed}
\end{figure}


\begin{figure*}
 \centering
 \includegraphics[trim=2.1cm 3.0cm 1cm 7.7cm, clip=true, width=0.70\textwidth]{figures/lores_panels_2750k}
 \caption{Top:  The low-resolution optical/near-IR spectrum of LkCa 4 obtained from Palomar/DBSP and APO/Triplespec on 30 December 2008 (black), compared to a synthetic spectrum of a two temperature photosphere (purple).  The inset shows that the 2750 K (red, 80\% fill factor) and 4100 K (blue, 20\% fill factor) components contribute equally to the near-IR spectrum, but the 4100 K component dominates the blue emission.  The synthetic spectrum is reddened by $A_V=0.4$ mag and scaled to the observed $J$-band spectrum. Bottom:  The low-resolution optical (left) and near-IR (right) spectrum of LkCa 4, compared with a 3900 K photosphere (blue), a 3500 K photosphere (red), and the two temperature photosphere (pink) that best fits the IGRINS spectrum.  The synthetic spectra are scaled separately to the optical spectrum at 0.75 $\mu$m and to the near-IR spectrum at 1.5 $\mu$m.  Warm photospheres accurately reproduce molecular bands at $0.7$ $\mu$m but fail to fit the spectral features at longer wavelengths.  Cooler photospheres predict molecular bands at $<0.7$ $\mu$m that are much deeper than observed.  The two temperature photosphere accurately fits spectral features in the optical and near-IR.}
 \label{fig:lores}
\end{figure*}



In the previous section, we established that the high resolution optical and near-IR spectra of LkCa 4 may be explained by a two-temperature photosphere.  In this section, we use testable predictions from this fit to demonstrate that the fit reasonably matches spectral features and their rotational modulation.  We adopt the best-fit two temperature model to the IGRINS spectrum, with components of 2750 K covering 80\% of the visible stellar surface and 4100 K covering 20\% of the stellar surface.  The IGRINS spectrum was obtained when LkCa 4 had an estimated brightness of $V=12.84$ mag.  The starspot and ambient temperatures and the filling factor are adopted without any adjustments to attempt to improve fits to the SED or broadband spectra.

Figures~\ref{fig:sed}-\ref{fig:lores} compares synthetic spectra from the two-component photosphere to the SED and flux-calibrated spectra of LkCa 4.  The only free parameters in this comparison are the luminosity and extinction, which are both scaled to match the spectrum (see \S 5.1-5.2).  The optical-IR SED is obtained from estimating photometry from \citet{grankin08} during the 2MASS epoch, with $V\sim12.61$ mag.  The SED also includes mid-IR photometry from Spitzer/IRAC \citep{hartmann05}, without adjusting for epoch.  In the spectroscopic comparison, some minor discrepancies between the optical and near-IR spectra may be introduced because the spot coverage changed in the $\sim 5$ hrs between observations ($\Delta V=0.08$ mag).  The synthetic spectrum is obtained from the Phoenix models, as in \S 3, and is extended beyond the longest wavelength (5 $\mu$m) for calculating the bolometric luminosity.

The synthetic models match the full spectrum reasonably well.  When scaled separately to the red-optical and near-IR spectra, the synthetic models match even better (bottom panels of Figure \ref{fig:lores}), while single temperature photospheres fail to reproduce large spectral features.   The $3900$ K component fits the TiO bands reasonably well at $<7400$ \AA\ but fails to reproduce molecular bands at longer wavelengths.  The $3500$ K component suffers from the opposite problem, yielding molecular bands at short wavelengths that are too deep.  A two temperature photosphere with the parameters from the best-fit calculated in \S 3 reproduces the TiO bands, the hump in the $H$-band, and the jump in flux at the long-wavelength end of the $J$-band.

These spectral features should be rotationally modulated as the filling factors of the hot and cool components change.    In \S 4.1, we first use our fit to the IGRINS spectrum to convert the $V$-band brightness to the filling factor of the two components.  We subsequently calculate the color changes expected based on this spot coverage and compare these results to historical data (\S 4.2), and confirm our basic approach by demonstrating that TiO band depths vary with rotation (\S 4.3).



\begin{figure*}
 \centering
 \includegraphics[trim=1.6cm 12.6cm 2.2cm 2.4cm, clip=true, width=0.60\textwidth]{figures/vband_spot_2750k}
\caption{The $V$-band magnitude in 2014--2015, converted into fill factor for the cool component.  The optical brightness depends mostly on the hot component.  If we fix a 75\% filling factor, as measured in the IGRINS spectrum, to $V=12.83$ at the time of the observation, then the $V$-band amplitude  corresponds to filling factors of 67--83\%.  The factor of $\sim 2$ change in visible surface area of the hot component, from 33\% to 17\%, is required to produce the $\Delta V=0.6$. }
\label{fig:vband_spot}
\end{figure*}
%\todo{Update with all available photometry or the derived multi-term fit} 



\begin{figure*}
 \centering
 \begin{tabular}{ll}
 \includegraphics[trim=2.6cm 13.0cm 2.8cm 3.2cm, clip=true, width=0.45\textwidth]{figures/bv_pec13} 
    &
   \includegraphics[trim=2.6cm 13.0cm 2.8cm 2.2cm, clip=true, width=0.45\textwidth]{figures/vr_pec13}
    \end{tabular}
\caption{The observed optical colors of LkCa 4 from \citet{grankin08}, compared with predictions.  The model is constructed instead by converting $V$-band brightness to a cool spot filling factor and subsequently calculating colors from main sequence colors and bolometric corrections of \citet{pecaut13} (purple lines).}
\label{fig:colors}
\end{figure*}


\subsection{The $V$-band lightcurve and spot coverage}\label{sec:rotSpot1}

The  $V$-band brightness reflects the instantaneous filling factor of the cool and hot components.  In the two-temperature photosphere, the $V$-band emission is dominated by the hotter component and is therefore a good proxy for the visible surface area of the hot component.  Figure~\ref{fig:vband_spot} shows the 2015--2016 ASAS-SN lightcurve, converted into a cool spot filling factor.  During this period, the brightest and faintest phase corresponds to cool component fill factors of $74\%$ and $86\%$, respectively.  Since 1992 the $V$ band photometry has been as bright as $12.3$ mag, which corresponds to a cool component fill factor of $65\%$, and as faint as $13.2$ mag, or a cool component fill factor of $87\%$.  This drastic change in visible area of the hot spot (35\% to 13\%) is needed to explain the full $\sim 1$ mag range in brightness, assuming no temperature change in either component.


\subsection{Rotational Modulation of Colors}\label{sec:rotSpot}

The relative contributions from the hot and cool components change as LkCa 4 rotates, leading to rotational modulation of colors.  We model this rotational modulation using the main sequence colors \citep[compiled by][]{pecaut13}, which should have less contribution from spots than  pre-main sequence colors.  The Cousins $R$ photometry from \citet{pecaut13} is converted to Johnson $R$ for comparison with the \citet{grankin08} photometry, following the color transformation prescribed by \citet{landolt83} and applied by Grankin et al.  Photometry from synthetic spectra produced by Phoenix and BT-Settl models cannot be used here because the predicted $B-V$ colors are much bluer than observed.  Offsets between model and observed colors are at least in part attributed to extinction and are discussed in \S 5.1.  

Limb darkening effects also become important if starspots are distributed in a finite number of large patches, not many small patches.  The variations in the mean projection angle cause confounding color trends in the photometric modulations.

The optical emission is dominated by the hotter component while both components contribute equally to the infrared emission (see inset in top panel of Figure \ref{fig:lores}).  Figure~\ref{fig:colors} demonstrates that the $B-V$ color from \citet{grankin08} does not depend on $V$, which is consistent with expectations for a single hot component with no contributions from the cooler component.  The standard deviation of 0.03 mag in $B-V$ color, which includes a $\sim 0.01$ mag uncertainty in both $V$ and $B$, indicates that the spot temperature is stable to $\lesssim 50$ K.  The stability of this temperature implies that this hotter component may be the ambient temperature of the photosphere.

The correlation between $V$ and $V-R$ indicates that the star becomes redder when the cool spot has a higher filling factor.  Our simple model predicts that the correlation should be much weaker than is measured.  Most likely, our two temperature model for the photosphere is overly simplistic, since the $V-R$ color could be reproduced with some contribution from intermediate temperature.

Rotational modulation is expected to lead to much smaller brightness changes at near-IR wavelengths than at optical wavelengths (see Table~\ref{tab:photrange}).  The smaller amplitude of near-IR brightness is also seen in a few spotted stars in the optical/IR monitoring of NGC 2264 \citep{cody14}, although those stars have a much smaller $V$ band amplitude than LkCa 4.




%Furlan:  K7, V-IC=1.92, E(V-I)=1.6 => 0.85 mag;  close enough to their 1.1 mag
%V-K = 4.18  ; MS colors:  3.16 (KH95) => Av=1.1 mag


%V-J:  2.8-3.8
%J-K:  0.9-1.0

\begin{table}
\caption{Predicted Photometric Range}
\centering
\label{tab:photrange}
\begin{tabular}{cccc}
\hline
Band & Low Spot & High Spot & Range\\
Cool Spot & 68\% & 83\% &\\
\hline
\multicolumn{4}{c}{Based on $V$, need to adapt for $A_V$}\\
$B$ & 13.80 & 14.48 & 0.68\\
$V$ & 12.56 & 13.23 & 0.67\\
$R$ & 11.53 & 12.17 & 0.64 \\
$I$ & 10.80 & 11.38 & 0.58\\
$J$ & 9.79 & 10.00 & 0.21\\
$K$ & 8.89 & 9.04 & 0.15\\
\hline
\end{tabular}
\end{table}



\subsection{Rotational Modulation of TiO Depth}\label{sec:RotTiO}

The depths of TiO and other molecular bands are common diagnostics of spectral type and therefore temperature in optical spectra \citep[e.g.][]{kirkpatrick91}.  In a 2-temperature photosphere cool enough for molecules to form, the depth of observed TiO bands will depend on the fractional coverage of the components.  The TiO band depths vary with spot coverage using 12 epochs of CFHT/ESPaDOnS spectra obtained over 14 days ($\sim 4$ rotation periods) by \citet{donati14}.

The spectra obtained near the estimated maximum and minimum optical brightness are shown in the top panel of Figure~\ref{fig:tiovar}.  Small but significant changes are seen in the TiO bands.  On the other hand, the blue spectrum does not change.  The TiO band depths in these spectra are measured from spectral indices as described in \citet{herczeg14}.  The TiO indices are also combined into a single TiO index by calculating the number of standard deviations each point is from the median value of the index and then averaging the standard deviation.  
% Could use a footnote to be more exact on the TiO indices.

Both the raw TiO 7140 index and this combined TiO index correlate strongly with the optical brightness.  In the spectral type scheme derived by \citet{herczeg14}, the change in TiO-7140 index corresponds to a range from M1.5 (3640 K) to M2.2 (3530 K).  These results are consistent with expectations from the rotational modulation of the hot spot.  The TiO bands trace the cool component.  When the hotter component has a larger filling fraction, the TiO bands are veiled by the hotter component and are therefore not as deep.


\begin{figure*}
 \centering
 \includegraphics[trim=3.2cm 4.0cm 2.6cm 8.3cm, clip=true, width=0.65\textwidth]{figures/lkca4_tio_forpaper}
 \caption{Variability in TiO bands measured with ESPaDOnS.  The $V-$band emission is estimated from fits to the ASAS-SN lightcurve obtained during the same period.  The bottom left panel shows a correlation between $V-$band magnitude and the TiO-7140 index, while the bottom right shows a similar correlation with the average of the TiO 6200, CaH 6800, and TiO 7600 indices.}
 \label{fig:tiovar}
\end{figure*}

\section{Placing LkCa 4 on the HR diagram}

%\todo{Some overview?}


\subsection{Extinction estimates to LkCa 4}

Multi-temperature photospheres may resolve some of the long-standing problems related to discrepancies between observed and model colors of young stars.  Low mass young stars are redder than expected in the near-IR, when compared with expectations from optical spectral types \citep[e.g.][]{tottle15}.  For LkCa 4, the excess $E(J-K)$ color leads to an extinction of $1.04$ mag for a 4200 K photosphere and \citet{pecaut13} colors.  This extinction is consistent with the color and SED analysis of \citet{furlan06}, which adopted a K7 spectral type, and is significantly higher than extinctions of $A_V=0.69$ measured from optical photometry \citep{kenyon95} and $0.35$ mag measured by comparing optical spectra to young spectral templates \citep{herczeg14}.  

An accurate extinction estimate from the full SED is challenging because of a strong dependence on the areal distribution of temperatures in the photosphere, which in our work is simplified to a two temperature fit.  However, isolating specific colors constrains the possible range of extinctions.  
The $B-V$ color is stable and not affected by cooler components, and may therefore be compared with the expected colors for the hotter (4100 K) temperature.  
The average $B-V=1.41$ mag color from \citet{grankin08} has a color excess $E(B-V)=0.09$ mag relative to the empirical color expected for a main sequence star and $E(B-V)=0.20$ mag for colors of pre-main sequence stars, based on the temperature-color tables of \citep{pecaut13}.  The main sequence colors are probably most 
appropriate in this analysis because they are more free of spots.  
The $B-V$ color 
excess leads to an extinction $A_V=0.28\pm0.15$ mag, or $0.62\pm0.15$ mag relative to 
pre-main sequence colors.   The listed uncertainties are calculated from extinction estimates relative to colors of 4000 and 4200 K photospheres.
The $V-R_J$ color excess leads to $A_V=0.48$ mag, which should be less affected by gravity (based on $V-I_C$ colors in Pecaut et al.) but is likely to be overestimated because the cool component contributes to the $R_J$-band emission, as seen in Fig.~\ref{fig:colors}.
In the near-IR, the estimated $V=12.61$ mag at the time of the 2MASS observations leads to a spot-weighted template (main sequence) color 
$J-K_s=0.87$ mag, with a range of $\sim 0.05$ mag for acceptable solutions with other 
temperatures and covering fractions. The observed $J-K=0.93$ mag leads to $A_V=0.35$ mag, 
with a range of $0.29$ mag to account for the range of other acceptable photospheric solutions.  
The template near-IR color is more robust than optical colors to differences 
in the main sequence/pre-main sequence temperature scales because the pre-main 
sequence $J-K$ color is redder at 4100 K and bluer at 2750 K.  Based on this 
analysis, an $A_V=0.3$ mag is adopted throughout this paper. 

The near-IR extinctions of LkCa 4, and perhaps other TTSs, will be overestimated if emission is generated from a multi-temperature spectrum but template colors are based on only the hotter component.  This mismatch likely explains why near-IR extinctions are often higher than optical extinctions for young stars \citep[see analysis in][]{herczeg14}.  Our results tentatively support the lower extinctions obtained at optical wavelengths, although the template colors used to calculate extinction are uncertain and will need revision.  Differences between main sequence and pre-main sequence temperature scales and colors in the \citet{pecaut13} tables are introduced by a combination of spots, in addition to any color differences caused by gravity differences.
Uncertainty in colors may be solved with self-consistent comparisons to young stars with similar spectral types, if spot coverage is also similar.  Discrepancies in colors, spectral type/effective temperatures, and extinctions for LkCa 4, and perhaps more broadly for TTSs, are dominated by ignoring spots and are not caused by fundamental problems in either photospheric models or in spectral type-effective temperature scales.




\subsection{Spot-corrected HR diagram placement}

\begin{figure*}
 \centering
 \includegraphics[trim=2.6cm 13.0cm 2.8cm 3.5cm, clip=true, width=0.65\textwidth]{figures/lkca4_hrdiag}
 \caption{Locations for LkCa 4 on an HR diagram, compared with models of pre-main sequence evolution calculated by \citet{baraffe15} with isochrones (black lines) and evolution models of a single mass (dashed blue lines) as marked.  The measured effective temperature and luminosity from this paper, based on the two-component fit and a median $V$-band magnitude, corresponds to the black asterisk.  The yellow shaded region corresponds to the range of apparent effective temperatures that are would be measured as the hot component rotates into and out of our view.  The blue circle corresponds to the measurement at blue-optical wavelengths by \citet{donati14}, the purple square corresponds to the measurement from low-resolution optical spectra, biased to TiO band strengths, by \citet[][biased to]{herczeg14}, while the red diamond corresponds to what we would measure from the K-band spectrum and 2MASS $J$-band magnitude.}
 \label{fig:hrdiag}
\end{figure*}

The spectrum of LkCa 4 is well reproduced by a cool component of $2750$ K covering 80\% of the surface and a $4100$ K component covering 20\% of the surface.   Scaling this composite spectrum for LkCa 4 to the observed $K$-band photometry yields $R=2.3$ $R_\odot$ and $\log L/L_\odot=0.17$, adopting $A_V=0.3$ mag and a distance of $131 \pm 2$ pc$^x$ based on the range of parallaxes of nearby Taurus members measured by \citet{torres12}\footnote{This distance is consistent with the Gaia distance of $130 \pm4$ pc measured to the nearby Taurus members V410 Tau, V773 Tau, and BP Tau \citep{gaia2016dr}.}.  The effective temperature of these two components is 3330 K, as calculated from the average surface flux at the median $V$ band magnitude.  Figure~\ref{fig:hrdiag} and Table~4 shows that this effective temperature is much cooler than previous measurements from optical spectra, which are clustered around 4000 K.  The luminosity does not change significantly because the broadband SED yields similar luminosities, independent of temperature.

The apparent effective temperature in any single epoch depends on the relative fractions of the hot and cool spot.  Decreasing the cool spot size to 74\% increases the effective temperature to 3440 K, while increasing the spot size to 86\% decreases to the effective temperature to 3200 K.  When accounting for the visible contributions of both hot and cool components, this wide range of possible observed effective temperatures at any single epoch underscores the difficulty in placing young stars on HR diagrams.  
Since the Hayashi tracks are nearly vertical at this temperature and age, the range in effective temperatures corresponds to a mass range from $\sim 0.2-0.3$ $M_\odot$.  Our IGRINS analysis is also consistent with a cool component of 3000 K instead of 2750 K, which would increase the effective temperature by $\sim 200$ K, and mass by 0.1 $M_\odot$.

The discrepancies in temperature are even larger when spectral type/temperature estimates focus on features only in either blue-optical or $H$ or $K$-band spectra, and may explain the severe differences in effective temperature between APOGEE and other (optical) methods \citep{cottaar14}.  

The INfrared Spectra of Young Nebulous Clusters (IN-SYNC) project \citep{cottaar14} employs APOGEE, which has a $15100-17000$ \AA\ bandwidth.  The long-wavelength cutoff prevents APOGEE from detecting the most conspicuous cool star emission we detected in IGRINS spectral orders $o = 100, 101, 104$ at $17075-17990 \;\AA$.  These IGRINS spectral orders that APOGEE misses can be thought of as a ``sweet spot'' where the cool star spectrum is putting out ample flux, has relatively conspicuous features, and is far-enough away from the $K-$band where problems like magnetic field-sensitive lines, and enhanced telluric absorption complicate the spectra.  Additionally, veiling from mid-IR excess in classical T-Tauri stars will be lessened in $H-$band.  

On the other hand, GAIA-ESO spectra \citep{frasca15} cover only a narrow wavelength around H$\alpha$ and would yield an effective temperature of 4000 K.   Spectral types and effective temperatures measured from TiO depths \citep{herczeg14} are lower than those from blue spectra but still overestimate the effective temperature.  Temperature uncertainties for young K and M dwarfs are larger than any formal uncertainties in individual measurements.

The improved characterization of the photospheric emission and radius of LkCa 4 would ideally lead to more accurate estimates of mass and age from pre-main sequence tracks.  However, our effective temperature and luminosity yield a mass of $0.2-0.3$ $M_\odot$ and an age of $\sim 0.5$ Myr in the \citet{baraffe15} evolutionary models\footnote{The \citet{baraffe15} evolutionary tracks were calculated using BT-Settl photospheric spectra \citep{allard14}, which may introduce some uncertainties when compared to results from the PHOENIX models used here.}.  This age is uncomfortably young and this mass is uncomfortably low.  LkCa 4 is not located deeply within a molecular cloud and is not associated with any nearby Class 0/1 stars, which would be expected for a 0.5 Myr star.  Dynamical measurements of masses of young stars with similar optical properties suggest that LkCa 4 should have a mass of $\sim 0.6-1.0$ M$_\odot$ \citep[e.g.][]{guilloteau14,czekala16,rizzuto16}.  Even though LkCa 4 is only a single data point, the inferred age and low mass may be evidence that strong magnetic fields are inhibiting convection in LkCa 4.  As a result, the surface is cooler and releases less radiation, slowing the contraction rate.  LkCa 4 then appears more luminous and cooler than expected for a star of its genuine age from evolutionary tracks that do not consider magnetic fields.   This shift to lower temperature may also help to resolve some of the age discrepancies between intermediate mass and low mass young stars \citep[e.g.][]{herczeg15}.  The corrected placement of spotted stars on the HR diagram reflects the accurate luminosity, radius, and effective temperature, but evolutionary models are not yet equipped to interpret such heavily spotted stars \citep{somers15}.

The IGRINS and ESPaDOnS spectra indicate indicate $\vsini\sim$ 28 km~s$^{-1}$ and a rotation period of 3.375 days. Combining these numbers gives $R\sin{i} \sim 1.87 R_{\odot}$. Our HR diagram analysis gives $R \sim 2.3R_{\odot}$. These numbers are consistent for $\sin{i} \sim 0.813$, or an inclination of about 35$^{\circ}$ from edge on. These values show broad consistency between rotational properties, spectral fitting values, and our interpretation of a tilted star revealing a circumpolar region with large areal coverage of starspots.

As seen in Figures \ref{fig:PhotTime} and \ref{fig:PhotPhase}, the areal coverage of spots appears to wax and wane secularly through the observing seasons of the last 31 years.  Notably, it appears to be in a relatively faint phase currently, 0.2 mag fainter in $V$ than it was through much of the late 1980's and early 1990's.  It is conceivable that LkCa 4 happens to be going through a short-lived episode of relatively high coverage of spots analogous to solar maximum.  The intensity and duration of such episodes would further confound the interpretation of its instantaneous position on the HR diagram.

\section{Discussion}

\subsection{Limitations of assumptions in the spectral inference methodology}

The spectral inference method assumed only two temperature components present in the photosphere.  In reality, starspots are probably described by a range of spot temperatures analogous to umbra, penumbra, and plages on the solar surface.  A model possessing three unique temperatures and two unique fill factors might better approximate the minor constituents of the emergent spectrum than our two temperature model can, but would always fit the data better and could lead to the possibility of overfitting.  The fit-quality of the existing two-temperature fits seems fairly strong through much of $H-$band and many $K-$band orders, so the contribution of minor photospheric constituents must be relatively low.  Interestingly, $K-$band has several spectral orders that converge towards hybrid temperatures in the vicinity of $\sim3500$ K.  Most of these orders are flagged as erroneous due to strong metal line spectral outliers or uncorrected calibration artifacts.  A few orders just show mediocre fits where no permutation of hot and cool two-temperature models could describe all of the variance in the spectrum.  These $K-$band orders could be examined for a penumbra-like temperature component.  We caution that systematic errors in the models would ultimately limit the interpretability of models with more than two temperature components.  Space-based high-precision planetary transit spectrophotometry could offer an avenue to measure the morphologies and temperature distributions of starspots for the subset of transiting exoplanet host stars or eclipsing binaries.

We assumed that the \PHOENIX\ synthetic spectra are a good representation of the observed IGRINS spectrum of LkCa 4, a young star that possesses a strong magnetic field \citep{donati14}.  Cool sunspots are associated with locally heightened magnetic fields on the sun, so synthetic spectral models employing negligible magnetic fields will fail to fit magnetic sensitive features.  Magnetic fields will have two effects on the emergent spectrum- line broadening attributable to the Zeeman effect, and pressure broadening attributable to seeing deeper into the photosphere in the line wings than in the line center.  The former effect will preferentially impact lines with high magnetic sensitivity \citep[\emph{e.g.}][]{johnskrull99,deen13}.  The latter effect could masquerade as an apparent shift in surface gravity, as seen in mixture models of spotted stars with empirical templates composed of dwarfs and giants \citep{oneal96}.  Our tacit assumptions have been that these line-broadening effects are 1) relatively small compared to the gross appearance and disappearance of spectral lines at disparate temperatures, 2) there exist many orders and lines that are not affected at all, and 3) the Gaussian process noise model fosters some resilience against unknown residual spectrum correlations induced by such under-fitting.  We see tentative evidence for physics outside of the pre-computed model grids in the poor fits to the observed spectra in $K-$band.  Strong metal lines demonstrate large departures in their best fit $\Z$, as the na\"{i}ve fitting strategy futilely attempts to minimize the residuals from Zeeman-broadened lines.  No pre-computed model grids including a range of surface magnetic fields are available at the wavelength range and spectral sampling needed to approach these questions within this spectral inference framework.

Uncertainties in the \PHOENIX\ synthetic spectra continuum opacity flow down to uncertainties in the absolute value of the fill factor.  Erroneous bulk opacities could be responsible for some of the scatter of fill factor distributions seen across different spectral orders, as in Figures \ref{fig:TwoTempResults} and \ref{fig:specPostageStamp}.  The large number of spectral orders over which the fill factors are averaged offers resilience against these anomalies.

\begin{figure*}
 \centering
\includegraphics[width=0.45\textwidth]{figures/v_v-r_Grankin} 
\caption{Grey points are all \citet{grankin08} photometric data, while black points are the subset of observations from the 1999 season.  The purple line represents a two component photometric model  described by \citet{grankin08}, comprising a photospheric temperature of 4040K (for K7), and a temperature of spotted area of 3030K (for an M5). In this model, the spot coverage of the observed hemisphere varies from $\sim36\%$ at maximum brightness, and $\sim82\%$ at minimum brightness.}
 \label{fig:grankin_vr}
\end{figure*}

Analyses of optical emission also suggest that a two-temperature model may be an oversimplification.  Figure \ref{fig:grankin_vr} shows that the $V-R$ colors from \citet{grankin} are better reproduced if the cooler component is $\sim 3050$ K, thereby contributing more flux at optical wavelengths.  The ZDI imaging of \citet{donati14} also indicate a complex structure of spots that are hotter than what is interpreted here as the ambient photosphere, while also reproducing the optical photometry.  Despite the need for more complicated emission structures to simultaneously reproduce all observations, the broadband SED, spectral shape, and rotational modulation of spectral indices are reasonably well fit with the two components presented here.  Moreover, the high and low-resolution near-IR spectra require a cool (2700--3000 K) component while the global photometry requires a large covering fraction for this cool component.  

The need for the cool component to have a large covering fraction and $<3000$ K is also demonstrated by the size of the H$_2$O band absorption between the $J$ and $H$ bands (see Figure~\ref{fig:h2ojump}).  Assuming that the hotter component is 4100 K, the H$_2$O opacity is only sufficient if $T_{spot}<3000$ K, with a high spot coverage.

\begin{figure*}
 \centering
 %\includegraphics[width=0.65\textwidth]{}
 \includegraphics[trim=2.6cm 13.0cm 2.3cm 2.5cm, clip=true, width=0.65\textwidth]{figures/jbandjump}
 \caption{The constraint on the size and temperature of the cool spot from the H$_2$O absorption band depth between $J$ and $H$-bands.  The contours show the flux ratio between 1.30 and 1.34 $\mu$m (shaded yellow region in inset plot), as produced by the Phoenix models.  The solid and hatched purple lines show the best fit and range of acceptable contours from the APO/TripleSpec spectrum, which was obtained when the cool spot fraction was $\sim 84$\%.  The range of acceptable contour is based on an estimate uncertainty of $\sim 2$\% in observed flux ratio, which is dominated by uncertainty in telluric correction of strong H$_2$O absorption. }
 \label{fig:h2ojump}
\end{figure*}

\subsection{Absolute lower limit of starspot filling factor}

A lower limit to the starspot filling factor can be set by the size of a black spot that can induce such large $\Delta V$.

Consider the following extremely conservative assumptions:  \begin{enumerate}
  \item The starspot emits \emph{zero flux}.  This statement is equivalent to the starspot being at \emph{absolute zero}.
  \item The stellar disk shows no starspots at maximum brightness.
  \item The starspot transits the center of the stellar disk.
\end{enumerate}

If \emph{any} these assumptions is relaxed, the minimum fill factor \emph{increases}.  

For LkCa 4 in 2015 the $\Delta V\sim 0.5$ mag amplitude corresponds to an absolute minimum fill factor $f>0.37$.  At least 37\% of the stellar is covered in starspots at minimum light.  In the 2004 season, LkCa 4 exhibited $\Delta V\sim 0.8$, corresponding to a minimum fill factor of $f=0.52$.

%\subsection{Absolute upper limit of starspot temperature}

%Consider relaxing assumption 1 from above to better match reality: a starspot is not at absolute zero temperature, it has a finite temperature and therefore emits a finite amount of radiation.  Increasing the temperature a little means that the starspot size has to increase to cause the same flux difference.  The starspot cannot occupy more than 100\% of the projected stellar hemisphere, so there is an upper limit for the temperature of the starspot that yields a given $\Delta V$.  The starspot temperature upper limit for $\Delta V=0.5$ is 3650 K.  In the 2004 season, the starspot temperature upper limit was 3410 K.  

%In reality, the star will have some patches of starspots that always face toward the observer, like polar spots, bands of spots, multiple finitie-sized patches of starspots, or homogenuously distributed small spots.  Dark spots on extreme latitudes on inclined stars might only pass over the limb of the stellar disk, reducing the diminution $V-$band flux for a given projected surface area.  All of these effect mean that real starspot filling factors are likely to be larger than the filling factor lower limit, and cooler than the temperature upper limit.


\subsection{Comparison to studies of ensembles}

%Introduce
The number of WTTS for which the areal filling factor of spots have been measured is relatively low.  Recently there have been many new estimates.

% Grankin
V 410 Tau shows up to 41\% coverage of spots \citep{petrov94}.  Many more measurements exist for evolved giants and subgiants, with typical fill factors in the range 15$-$48\% \citep{berdyugina05}.  The prototypical BY Dra source is consistent with 60\% coverage fraction of moderately cool ($\Delta \teff =400$) penumbra \citep{chugainov76}.  By these comparisons, LkCa 4 seems on the high end of filling factor.

There are several WTTS in the Taurus-Auriga star-forming region that demonstrate the highest amplitudes of periodic light variations, achieving $0.4-0.8$ mag in the $V-$band: LkCa 4, V410 Tau, V836 Tau, LkCa 7, V827 Tau, V830 Tau, and V819 Tau \citep{grankin08}.  Such large amplitudes of the light variation indicate the existence of very extended spotted regions on the stellar surface. To estimate the total area and mean temperature of the spots in the visible stellar hemisphere Grankin used a simple non-parametric model for analysis of V410 Tau, V819 Tau, V827 Tau, V830 Tau, and V836 Tau \citep{grankin98,grankin99}. For the stars with the highest light curve amplitudes, this model shows that the spotted regions cover from 17 to 73\% of the visible stellar hemisphere and that the mean temperature of the spots is $500-1400$ K lower than the ambient photosphere.  An analysis of the calculated spot parameters of V410 Tau revealed that there is a real correlation between the amplitude of periodicity and the spot distribution
nonuniformity, which is defined as the difference of spot coverage in the visible hemispheres of the star at maximum and minimum light. As the amplitude increases from 0.39 to 0.63mag, the degree of nonuniformity in the spot distribution increases from 21 to 37\%. Besides the high amplitude of variability, these objects show the phenomenon of long-term stability of a brightness minimum in the interval from 5 to 19 years.  To all appearances, this group of stars has the similar nature of stellar activity which
should be investigated in the future.

% Covey
The Pleiades serves as an older age sample of young stars possessing evidence for large coverage fraction of starspots.  For example, anomalous colors of K dwarfs in the Pleiades can be explained by $\sim$50\% fill factors of spots \citep{stauffer03}.  Recent photometric monitoring of the Pleiades indicates that the anomalous colors scale with rotation rate, strengthing the case for magnetic-field induced starspots as the cause of the anomaly \citep{covey16}.  However, the \citet{covey16} Pleiades rotation rates do not scale with the $\Delta V$ photometric amplitude at the precision of the PTF data, suggesting that while the overall starspot coverage increases with rotation rate, the longitudinally asymmetric component does not.  LkCa 4 represents an example of such an architecture, where more than 50\% of the stellar surface is covered in spots, regardless of the phase of observation.  The longitudinally symmetric component evades photometric modulation detection, but contributes to otherwise anomalous color offsets.

% Venuti
WTTSs observed towards NGC2264 show photometric variations similar to--but of lower average amplitude than-- LkCa 4 \citep{venuti15}.  Modeling of polychromatic photometric monitoring of the locus of WTTSs indicates a lower limit to the true coverage fraction $f_{\mathrm{eq}}\sim10-30\%$ and a difference $\teffa-\teffb\sim500$ K.

% Fang
%\textbf{Insert Fang info here.}



\subsection{Interpretation of the large filling factor of cool starspots}
We have shown that LkCa 4 exposes between 74\% and 86\% cool photosphere towards our viewing direction in 2015-2016, the most recent season for which there is data.  In previous seasons, like Fall 2004 (\emph{c.f.} Figures \ref{fig:PhotPhase} and \ref{fig:vband_spot}), the filling factor of cool spots could have exceeded 90\%.  This exceptionally large fraction of surface area covered by spots defies the conventional wisdom that starspots are the minority constituent of the stellar surface.  The evidence for this derived fill factor has been shown in the SED, high-resolution spectral inference, direct spectral detection, and TiO variability modelling.  Perhaps the strongest evidence for the large fill factor of cool photosphere is the large amplitude of $V-$band modulation, which cannot be explained with small spots, even if a spot temperature of absolute zero is assigned to the cool regions.

One conceivable interpretation of the 74\% to 86\% coverage fraction of cool photosphere is a stellar axis inclined towards our observing direction, with a large circum-polar spot covering much of the observable hemisphere, as described by \citet{donati14}.  The ambient hot photosphere would then be close to the equator, which would be limb-darkened due to projection effects.  Such limb darkening would cause a non-linear transformation between our observed fill factor $f_{\Omega}$, and the surface areal coverage fraction of spots.  It is conceivable, and probably likely, that the value we report for $f_{\Omega}$ is an overestimate of the surface area actually covered by spots.  In this circum-polar spot scenario, local fingers of cool spots could reach down to near the equator, entering into and out-of the field of view and giving rise to the large amplitude photometric variation seen on LkCa 4.  

The starspot geometry is also consistent with morphologies observed in interferometric maps of $\zeta$ Andromedae: large areal coverage of permanent polar spots, and smaller, transient networks of varying darkness which thread across the whole surface \citep{roettenbacher16}.

%\subsection{Possibility of third temperature or multiple temperatures}


\subsection{Is \name an extreme source?}
The large amplitude of cyclical optical variability of LkCa 4 exceeds all other counterparts from long-term monitoring of WTTSs \citep{grankin08}.  From a statistical perspective, it might not be a surprise that the source with the largest photometric variability also posessess among the largest coverage fraction of cool spots.



Despite the evidence that LkCa 4 is an outlier, spots are likely broadly important in estimating the properties of young low-mass stars. As an example, the well-studied CTTS TW Hya has a $\sim 4000$ K photosphere measured from high resolution optical spectra \citep[e.g.][]{yang05}.  However, the near-IR spectrum is consistent with $\sim M2$, when compared with main sequence dwarf stars \citep{vacca11}.  \citet{debes13} reconciled this discrepancy by invoking spots, although the spectral contribution from accretion and disk continuum emission was uncertain.  \citet{mcclure13} found that the TW Hya spectrum was well reproduced with near-IR spectra of stars with similar ($\sim$K7) optical spectral type, as long as the comparison spectra were young.   \citet{herczeg14} found an intermediate spectral type for TW Hya and other TTSs by focusing on TiO bands at redder wavelengths than previous spectra, and also by using young stars as photospheric templates.  The mismatch between near-IR spectra of dwarf stars and young stars, even with similar optical spectral types, is likely explained by the prevalence and importance of cool spots in low mass stars.

\subsection{Evidence for intra-spectrum RV jitter}
Starspots cause significant jitter in radial velocity (RV) surveys for young planets \citep[e.g.][]{donati14, robertson14}.  The two-temperature model assumed that the hot and cool components shared the same $\vsini$ and $v_z$.  This assumption is accurate for either homogeneously distributed spots, azimuthally symmetric bands of spots, or polar spots.  LkCa 4 has clear optical line-profile variations \citep{nguyen12, donati14}.  We should be able to detect \emph{anticorrelation} between the line center positions of spectral lines arising from ambient photosphere, and lines arising from cool starspots, since there is a zero-sum competition for solid-angle on the recessional and advancing stellar limbs.  We found tentative, albeit weak, evidence for such anticorrelation in $v_z$ across spectral orders dominated by cool photosphere and those dominated by hot photosphere.  For example, the cool-photosphere dominated spectral orders $o = 100, 102, 104$ show the largest median $v_z\sim14.5-17.0$ km/s, while the other orders that possess a mix of primarily hot- and some cool- photosphere features demonstrate $v_z\sim9-13$ km/s.  An independent analysis of the RV of the IGRINS spectrum yields a barycentric RV of $14.6\pm0.2$ km~s$^{-1}$.  The direct emission of starspots could constitute a heretofore unaccounted source of correlations in conventional RV signal processing.

Radial velocity (RV) planet searches in the infrared--where the modulation of RV signals is lessened \citep[\emph{e.g.}][]{prato08}-- could potentially reduce RV jitter by characterizing and constraining the effect of direct emission from starspots, in addition to the flux deficits imbued in optical line profiles.  Direct spectral emission from cool starspots is likely much weaker in sources with less starspot coverage than LkCa 4.  Still, the characterization and modeling of direct emission from starspots should, in principle, decrease RV jitter attributable to starspots in the near-IR spectra of any spotted star as demands on RV precision increase.


\section{Conclusions}
This paper has presented several complementary constraints on the photospheric properties of the weak-lined T-Tauri star LkCa 4, showing that this heavily spotted star was previously misplaced on the pre-main sequence HR diagram.  New results from \citet{fang2016} show that large starspot coverage fractions of $50\%$ are common among stars in the Pleiades.  LkCa 4 breaks the apparent trend of a ceiling at 50\% coverage fraction of cool starspots, though it is not yet clear how to classify stars for which the minority of the surface is purportedly the ambient photosphere.

Coupled with the results of \citet{fang2016} and \citet{covey16}, the results shown here suggest that the contribution of starspots to pre-main sequence stellar evolution have been systematically underestimated.  Large parts of the stellar surface are covered in spots that do not induce photometric modulation, either from circumpolar regions that always face towards our line-of-sight, longitudinally symmetric bands, or isotropic small spots.

The spectral inference technique described in this paper lays the foundation for future studies that seek accurate constraints on multiple photospheric components.  The near-IR has been shown to be the most sensitive region to search for cool emission from starspots.  Starspots and disk or accretion veiling can be modeled in this framework, and combining the two will enable studies of the large number of classical T-Tauri stars.  Applying this technique to large ensembles of T-Tauri stars across a wide range of mass and age will offer an observational picture of starspot evolution that can revise the current understanding of pre-main sequence stellar evolution.  


\clearpage
\pagebreak


\appendix

\section{Methodological details}
\label{methods-details}

\subsection{Addressing absolute flux calibration}
If the stellar models are in absolute flux units, our mixture model is complete as stated in Equation \ref{eqn:mix_M}.  However \iancze\ employ \emph{standardized} $\flam$ when constructing the forward model $\vM$:


\begin{eqnarray} \label{eqn:normalization}
\bar \flam = \frac{\flam}{\int_{0}^{\infty} \flam d\lambda} = \frac{\flam}{f}
\end{eqnarray}

The reason for standardizing the flux is a matter of practicality: the number of PCA eigenspectra components in the spectral emulator scales steeply with the pixel-to-pixel variance of $f_{\lambda}(\{\vt_{\ast}\}^\textrm{grid})$, increasing the computational cost of spectral emulation.  The choice to standardize fluxes makes no difference for modeling a single photospheric component.  But for two-component photosphere models, the relative flux of the two model spectra needs to be accounted for to get an accurate estimate of the areal coverage fraction of the cool spots, $f_{\Omega} \equiv \Omega_{\mathrm{spot}}/(\Omega_{\mathrm{amb}}+\Omega_{\mathrm{spot}})$.  So we scale the mixture model in the following way:

\begin{eqnarray} \label{eqn:norm_scaling}
f_{\lambda, \mathrm{mix}} &=& f_{\mathrm{amb}} \bar f_{\lambda, \mathrm{amb}} \times \Omega_{\mathrm{amb}} + f_{\mathrm{spot}} \bar f_{\lambda, \mathrm{spot}} \times \Omega_{\mathrm{spot}} \\
q &=& q(\vt_{\ast})\equiv \frac{f_{\mathrm{spot}}(\teffb, ...)}{f_{\mathrm{amb}}(\teffa, ...)} \\
f_{\lambda, \mathrm{mix}}^{\prime} &=& \bar f_{\lambda, \mathrm{amb}} \times \Omega_{\mathrm{amb}} + q \bar f_{\lambda, \mathrm{spot}} \times \Omega_{\mathrm{spot}}
\end{eqnarray}

where the prime symbol in the final line indicates a re-standarized mixture model flux, where the relative fluxes of the model components are now correctly scaled.

We are then tasked with computing an estimator, $\hat f(\vt_{\ast})$, for estimating the scale factor $q$ in-between model gridpoints.  One robust approach would be to follow \iancze\ by training a Gaussian process on the $f(\{\vt_{\ast}\}^\textrm{grid})$.  Instead, we simply linearly interpolated between the model grid-points.  Interpolation can cause pile-up near model grid-points, as noted in \citet{cottaar14}, which motivated the spectral emulation procedure in \iancze.  We assume the interpolation of $\hat f$ is smooth enough that we will not see such pileups and if even we did see pileups, they would mostly be discernable as kinks in the distribution of samples in the starspot coverage fraction $f_{\Omega}$.  One drawback of our estimator method compared to the Gaussian Process regression method is that we do not propagate the uncertainty associated with the absolute flux ratio interpolation into our estimate of $f_{\Omega}$.  We assume this uncertainty is relatively small and can be ignored.

The mixture model is a linear operation with all the same stellar extrinsic parameters, so we can re-use all the same post-processed eigenspectra $\widetilde{\mathbf{\Xi}}$, mean spectrum $\widetilde{\xi}_\mu$, and variance spectrum $\widetilde{\xi}_\sigma$, with the tildes representing all post processing \emph{except} the $\Omega$ scaling.  We calulate \emph{two} sets of eigenspectra weights $\mathbf{w}_{\in (\mathrm{amb}, \mathrm{spot})}$, and their associated mean and covariances following the Appendix of \iancze, and yielding:

\begin{equation}
  \mathsf{M}_{\mathrm{mix}}^\prime = \Omega_{\mathrm{amb}} (\widetilde{\xi}_\mu + \mathbf{X} \mathbf{\mu}_{\mathbf{w}, \mathrm{amb}}) + q \Omega_{\mathrm{spot}} (\widetilde{\xi}_\mu + \mathbf{X} \mathbf{\mu}_{\mathbf{w}, \mathrm{spot}})
\end{equation}

\begin{equation}
  \mathsf{C}_{\mathrm{mix}}^\prime = \Omega_{\mathrm{amb}}^2 \mathbf{X} \mathbf{\Sigma}_\mathbf{w, \mathrm{amb}} \mathbf{X}^T + q \Omega_{\mathrm{spot}}^2 \mathbf{X} \mathbf{\Sigma}_\mathbf{w, \mathrm{spot}} \mathbf{X}^T + \mathsf{C}
  \label{eqn:modC}
\end{equation}

%\todo{Verify that the final term possesses a $q$ and not a $q^2$ term.}


\section{Examples of full spectrum fitting}

Figures \ref{fig:Hband3x7} and \ref{fig:Kband3x7} show spectra from 42 IGRINS orders.


\begin{figure*}
 \centering
 \includegraphics[width=0.95\textwidth]{figures/H_band_spectra_3x7}
 \caption{IGRINS Orders $122-100$, with panels arranged with the shortest wavelength in the upper left corner with central wavelenth decreasing toward the bottom of the leftmost column, then decreasing through the subsequent columns.  The $y-$axis is on a logarithmic scale.  The red line is the cool photosphere while the blue line is the hot photosphere.  The purple line is the composite mixture model.}
 \label{fig:Hband3x7}
\end{figure*}

\begin{figure*}
 \centering
 \includegraphics[width=0.95\textwidth]{figures/K_band_spectra_3x7}
 \caption{IGRINS Orders $93-73$, with the same layout, ordering, and colors as Figure \ref{fig:Hband3x7}. The $y-$axis is on a logarithmic scale.}
 \label{fig:Kband3x7}
\end{figure*}



\section{Full table of IGRINS best fits}

Table \ref{tbl_order_results} lists all the IGRINS spectral order results.
\LongTables
%%%%%%%%%%%%%%%%%%%%%%%%%%%%%%%%%%%%%%%%
% TABLE - History of LkCa4
%%%%%%%%%%%%%%%%%%%%%%%%%%%%%%%%%%%%%%%%
\begin{deluxetable}{ccccccc}

\tabcolsep=0.11cm
%\rotate
\tabletypesize{\footnotesize}
\tablecaption{Multi-order results from approaches 12 and 22\label{tbl_order_results}}
\tablewidth{0pt}
\tablehead{
& & & Model 12 & \multicolumn{3}{c}{Model 22} \\
\cline{4-5} \cline{5-7} 
\colhead{Order} &
\colhead{Instrument} &
\colhead{$\lambda_1-\lambda_2$} &
\colhead{$T_{eff}$} &
\colhead{$T_{eff, 1}$} &
\colhead{$T_{eff, 2}$} &
\colhead{$c$} \\
\colhead{} &
\colhead{} &
\colhead{$\mu$m} &
\colhead{K} &
\colhead{K} &
\colhead{K} &
\colhead{}
}
\startdata
 m115 & IGRINS & 16543-16544 & 4100 & 4100 & 3100 & 0.70 \\
 m110 & IGRINS & x-x & x & x & x & x \\
\enddata

\tablecomments{See ref blah for more info.}
%\tablerefs{}

%\end{deluxetable*}
\end{deluxetable}

\section{Previous work}

Table \ref{tbl_history} lists measurements of \name from previous studies.
%%%%%%%%%%%%%%%%%%%%%%%%%%%%%%%%%%%%%%%%
% TABLE - History of LkCa4
%%%%%%%%%%%%%%%%%%%%%%%%%%%%%%%%%%%%%%%%
%\begin{deluxetable*}{lccccccccc}
\begin{deluxetable}{p{4cm}ccccccccc}

\tabcolsep=0.11cm
%\rotate
\tabletypesize{\footnotesize}
\tablecaption{Previous studies of LkCa4\label{tbl_history}}
\tablewidth{0pt}
\tablehead{
\colhead{Ref} &
\colhead{Band(s)} &
\colhead{Resolution} &
\colhead{Classification} &
\colhead{$T_{eff}$} &
\colhead{$\log{g}$} &
\colhead{$A_V$} &
\colhead{[Fe/H]} &
\colhead{$v\sin{i}$} &
\colhead{$v_{z}$} \\
\colhead{} &
\colhead{} &
\colhead{} &
\colhead{} &
\colhead{K} &
\colhead{} &
\colhead{} &
\colhead{km/s} &
\colhead{km/s} &
}
\startdata
 Junk et al. & $V$ & 10000 & - &4000 & 3.9 & 0.0 & 0.0 & 20.1 & 15 \\
 \citet{1986AJ.....91..575H} & $V$ &  & K7 V & - & - & - & - & 26.1$\pm$2.4 & +13$\pm$4 \\
 \citet{1987AJ.....93..907H} & $U$ & $<$2 km/s & T-Tauri & - & - & - & - & 26.1$\pm$2.4 & +16.9$\pm$2.6 \\
 \citet{1988AJ.....96..777D} & $V$ & 13$\AA$ & Me & - & - & - & - & - & - \\
 \citet{1989AJ.....97.1451S} & $V$ & - & K7:V & - & - & 0.95 & - & - & - \\
 \citet{1989AJ.....98.1444S} & $V$ & 0.3$\AA$ & K7:V & - & - & - & - & - & - \\
 \citet{1994ApJ...424..237S} & $V$ & - & K7 & 4000 & - & 1.25 & - & - & - \\
% \citet{1994A&A...282..503M} & $V$ & ? & K7 & 4130 & 3.65 & - & - & - & - \\	
 \citet{1995ApJS..101..117K} & ? & ? & K7 & 4060 & 3.65 & 0.69 & - & - & - \\
 \citet{1995ApJ...452..736H} & ? & ? & K7 & 4000 & - & 0.68 & - & - & - \\
\enddata

\tablecomments{Some values are not original, see references to trace to original source.}
%\tablerefs{}

%\end{deluxetable*}
\end{deluxetable}



\acknowledgements

MG-S and GJH are supported by general grant 11473005 awarded by the National Science Foundation of China.   The ESPaDOnS observations are supported by the contribution to the MaTYSSE Large Project on CFHT obtained through the Telescope Access Program (TAP), which has been funded by the ``the Strategic Priority Research Program---The Emergence of Cosmological Structures'' of the Chinese Academy of Sciences (Grant No.11 XDB09000000) and the Special Fund for Astronomy from the Ministry of Finance. 
TW-SH is supported by the DOE Computational Science Graduate Fellowship, grant number DE-FG02-97ER25308.

This work used the Immersion Grating Infrared Spectrograph (IGRINS) that was developed under a collaboration between the University of Texas at Austin and the Korea Astronomy and Space Science Institute (KASI) with the financial support of the US National Science Foundation under grant ASTR1229522, of the University of Texas at Austin, and of the Korean GMT Project of KASI.

This research has made use of NASA's Astrophysics Data System.  The research is based on data from the OMC Archive at CAB (INTA-CSIC), pre-processed by ISDC.
We acknowledge with thanks the variable star observations from the AAVSO International Database contributed by observers worldwide and used in this research.
This publication makes use of data products from the Two Micron All Sky Survey, which is a joint project of the University of Massachusetts and the Infrared Processing and Analysis Center/California Institute of Technology, funded by the National Aeronautics and Space Administration and the National Science Foundation.


{\it Facilities:} \facility{Smith (IGRINS)}, \facility{AAVSO}, \facility{CFHT (ESPaDOnS)}, \facility{INTEGRAL (OMC)}, \facility{ASAS}, \facility{CrAO:1.25m}, \facility{ARC (TripleSpec)}, \facility{Hale (DBSP)}, \facility{Gaia}

{\it Software: } 
 \project{pandas} \citep{mckinney10},
 \project{emcee} \citep{foreman13},
 \project{matplotlib} \citep{hunter07},
 \project{numpy} \citep{vanderwalt11},
 \project{scipy} \citep{jones01},
 \project{ipython} \citep{perez07},
 \project{gatspy} \citep{JakeVanderplas2015},
 \project{starfish} \citep{czekala15},
 \project{seaborn} \citep{waskom14}
%\software{%
% \project{pandas} \citep{mckinney10}
%    \project{emcee} \citep{foreman13},
% \project{matplotlib} \citep{hunter07},
% \project{numpy} \citep{vanderwalt11},
% \project{scipy} \citep{jones01},
% \project{ipython} \citep{perez07},
% \project{gatspy} \citep{JakeVanderplas2015},
% \project{starfish} \citep{czekala15}}.

\clearpage

\bibliographystyle{apj}
\bibliography{ms}

\end{document}


