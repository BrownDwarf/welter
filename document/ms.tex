%\documentclass[iop,revtex4]{emulateapj}% change onecolumn to iop for fancy, iop to twocolumn for manuscript
\documentclass[onecolumn]{emulateapj}% change onecolumn to iop for fancy, iop to onecolumn for manuscript
%\documentclass[12pt,preprint]{aastex}

%\usepackage{lineno}
%\usepackage{blindtext}
%\linenumbers

\let\pwiflocal=\iffalse \let\pwifjournal=\iffalse
%From: http://arxiv.org/format/1512.00483
%\input{setup}
\usepackage{enumerate}
\usepackage{amsmath,amssymb}
\usepackage{bm}
\usepackage{color}
\usepackage[utf8]{inputenc}

%% For anyone who downloaded my source file from arxiv:
%% I stole most of this setup.tex from a paper by Peter .K.G. Williams, but I made a bunch of edits to satisfy my own needs. You might check his paper out (http://arxiv.org/abs/1409.4411) for the original source file or contact him if you have any questions, since I don't really understand how some of these things work. 
%One cool thing it does is you can define an object, just that when someone clicks on the pdf it will link to simbad. I could never quite get this to work, probably because you have to get the text exactly right and my motivation for getting it to work was not super high. 


% basic packages
\usepackage{amsmath,amssymb}
\usepackage[breaklinks,colorlinks,urlcolor=blue,citecolor=blue,linkcolor=blue]{hyperref}
\usepackage{epsfig}    
\usepackage{graphicx}    
\usepackage{lineno}
\usepackage{natbib}
\usepackage{bigints}
\usepackage[outdir=./]{epstopdf}



% font stuff
\usepackage[T1]{fontenc}
\pwifjournal\else
  \usepackage{microtype}
\fi


% emulateapj has overly conservative figure widths, I think because some
% people's figures don't have good margins. Override.
\pwifjournal\else
  \makeatletter
  \renewcommand\plotone[1]{%
    \centering \leavevmode \setlength{\plot@width}{0.95\linewidth}
    \includegraphics[width={\eps@scaling\plot@width}]{#1}%
  }%
  \makeatother
\fi


\makeatletter

\newcommand\@simpfx{http://simbad.u-strasbg.fr/simbad/sim-id?Ident=}

\newcommand\MakeObj[4][\@empty]{% [shortname]{ident}{url-escaped}{formalname}
  \pwifjournal%
    \expandafter\newcommand\csname pkgwobj@c@#2\endcsname[1]{\protect\object[#4]{##1}}%
  \else%
    \expandafter\newcommand\csname pkgwobj@c@#2\endcsname[1]{\href{\@simpfx #3}{##1}}%
  \fi%
  \expandafter\newcommand\csname pkgwobj@f#2\endcsname{#4}%
  \ifx\@empty#1%
    \expandafter\newcommand\csname pkgwobj@s#2\endcsname{#4}%
  \else%
    \expandafter\newcommand\csname pkgwobj@s#2\endcsname{#1}%
  \fi}%

\newcommand\MakeTrunc[2]{% {ident}{truncname}
  \expandafter\newcommand\csname pkgwobj@t#1\endcsname{#2}}%

\newcommand{\obj}[1]{%
  \expandafter\ifx\csname pkgwobj@c@#1\endcsname\relax%
    \textbf{[unknown object!]}%
  \else%
    \csname pkgwobj@c@#1\endcsname{\csname pkgwobj@s#1\endcsname}%
  \fi}
\newcommand{\objf}[1]{%
  \expandafter\ifx\csname pkgwobj@c@#1\endcsname\relax%
    \textbf{[unknown object!]}%
  \else%
    \csname pkgwobj@c@#1\endcsname{\csname pkgwobj@f#1\endcsname}%
  \fi}
\newcommand{\objt}[1]{%
  \expandafter\ifx\csname pkgwobj@c@#1\endcsname\relax%
    \textbf{[unknown object!]}%
  \else%
    \csname pkgwobj@c@#1\endcsname{\csname pkgwobj@t#1\endcsname}%
  \fi}

\makeatother


% Evil magic to patch natbib to only highlight the year paper refs, not the
% authors too; as seen in ApJ. From
% http://tex.stackexchange.com/questions/23227/.

\pwifjournal\else
  \usepackage{etoolbox}
  \makeatletter
  \patchcmd{\NAT@citex}
    {\@citea\NAT@hyper@{%
       \NAT@nmfmt{\NAT@nm}%
       \hyper@natlinkbreak{\NAT@aysep\NAT@spacechar}{\@citeb\@extra@b@citeb}%
       \NAT@date}}
    {\@citea\NAT@nmfmt{\NAT@nm}%
     \NAT@aysep\NAT@spacechar\NAT@hyper@{\NAT@date}}{}{}
  \patchcmd{\NAT@citex}
    {\@citea\NAT@hyper@{%
       \NAT@nmfmt{\NAT@nm}%
       \hyper@natlinkbreak{\NAT@spacechar\NAT@@open\if*#1*\else#1\NAT@spacechar\fi}%
         {\@citeb\@extra@b@citeb}%
       \NAT@date}}
    {\@citea\NAT@nmfmt{\NAT@nm}%
     \NAT@spacechar\NAT@@open\if*#1*\else#1\NAT@spacechar\fi\NAT@hyper@{\NAT@date}}
    {}{}
  \makeatother
\fi

\newcommand{\prob}{{\rm prob}}
\newcommand{\qN}{\{q_i\}_{i=1}^N}
\newcommand{\qM}{\{q_{im}\}_{i=1,m=0}^{N,M}}
\newcommand{\yN}{\{y_i\}_{i=1}^N}

\newcommand{\kms}{ \textrm{km s}^{-1} }

\newcommand{\vM}{\mathsf{M}}
\newcommand{\vD}{\mathsf{D}}
\newcommand{\vR}{\mathsf{R}}
\newcommand{\vC}{\mathsf{C}}
\newcommand{\fM}{ \vec{{\bm M}}}
\newcommand{\fMi}{M_i}
\newcommand{\fD}{ \vec{{\bm D}}}
\newcommand{\fDi}{D_i}
\newcommand{\fR}{ {\bm R}}
\newcommand{\dd}{\,{\rm d}}
\newcommand{\trans}{\mathsf{T}}
\newcommand{\teff}{T_\textrm{eff}}
\newcommand{\logg}{\log g}
\newcommand{\Z}{[{\rm Fe}/{\rm H}]}
\newcommand{\A}{[\alpha/{\rm Fe}]}
\newcommand{\vsini}{v \sin i}
\newcommand{\matern}{Mat\'{e}rn}
\newcommand{\HK}{$\textrm{H}_2$O-K2}
\newcommand{\cc}[2]{c_{#2}^{(#1)}} 

\newcommand{\flam}{f_\lambda}
\newcommand{\vt}{ {\bm \theta}}
\newcommand{\vT}{ {\bm \Theta}}
\newcommand{\vp}{ {\bm \phi}}
\newcommand{\vP}{ {\bm \Phi}}
\newcommand{\cheb}{ \vp_{\mathsf{P}}}
\newcommand{\chebi}[1]{ \vp_{\textrm{Cheb}_{#1}}}
\newcommand{\Cheb}{ \vP_{\textrm{Cheb}}}
\newcommand{\Chebi}[1]{ \vP_{\textrm{Cheb}_{\ne #1}}} 
\newcommand{\cov}{ \vp_{\mathsf{C}}}
\newcommand{\covi}[1]{ \vp_{\textrm{cov}_{#1}}} 
\newcommand{\Cov}{ \vP_{\textrm{cov}}}
\newcommand{\Covi}[1]{ \vP_{\textrm{cov}_{\ne #1}}} 

\newcommand{\allParameters}{\vT} 
\newcommand{\nuisanceParameters}{\vP} 

\newcommand{\KK}{\mathcal{K}}
\newcommand{\Kglobal}{\KK^{\textrm{G}}}
\newcommand{\Klocal}{\KK^{\textrm{L}}}

\newcommand{\Gl}{Gl\,51}
\newcommand{\PHOENIX}{{\sc Phoenix}}

% Appendix commands
\newcommand{\wg}{\mathbf{w}^\textrm{grid}}
\newcommand{\wgh}{\hat{\mathbf{w}}^\textrm{grid}}

\newcommand{\Sg}{\mathbf{\Sigma}^\textrm{grid}}


\newcommand{\todo}[1]{ \textcolor{blue}{\\TODO: #1}}
\newcommand{\comm}[1]{ \textcolor{red}{SA: #1}}
\newcommand{\hili}[1]{ \textcolor{green}{#1}}
\newcommand{\ctext}[1]{ \textcolor{blue}{\% #1}}


%  From Peter Williams and Andy Mann again:
\newcommand{\um}{$\mu$m}

\providecommand{\eprint}[1]{\href{http://arxiv.org/abs/#1}{#1}}
\providecommand{\adsurl}[1]{\href{#1}{ADS}}
\newcommand{\name}{LkCa4 }
%\def\vsini{$v\sin{i_*}$}

\slugcomment{In preparation}

\shorttitle{\name IGRINS spectroscopy}

\shortauthors{Gully-Santiago et al.}

\bibliographystyle{yahapj}

\begin{document}
 
\title{IGRINS spectra of \name}

\author{Michael A. Gully-Santiago,\altaffilmark{1} Greg Herczeg,\altaffilmark{1} et al.}


\altaffiltext{1}{Kavli Institute for Astronomy and Astrophysics, Beijing, China}

\begin{abstract}
We interpret the high resolution near-IR spectra of \name.  We used the instrument IGRINS.
\end{abstract}

\keywords{stars: fundamental parameters --- stars: individual (\name) ---  stars: low-mass -- stars: statistics}

\maketitle

\section{Introduction}\label{sec:intro}


For a single presumably coeval stellar cluster, pre main-sequence HR diagrams exhibit large ($\delta t_{\ast} \sim t_{\ast}$) apparent spreads in age \citep[e.g.][]{2011A&A...534A..83R}. The large spreads in these pre main-sequence HR diagrams is controversial: some take it as evidence for intrinsic age spreads, some take it as evidence for non static accretion history \citep{2009ApJ...702L..27B, 2010ARA&A..48..581S}.  

Still others take the large apparent age spreads in pre main-sequence HR diagrams as evidence for the limited physics included in the pre main-sequence evolutionary model isochrones.  Most modern evolutionary models do not yet include the effects of starspots or magnetic fields, despite observations betraying their presence, \emph{e.g.} optical monitoring and Zeeman doppler imaging \citep{2008A&A...479..827G,2014MNRAS.444.3220D}.  

Meanwhile, sunspots are probably responsible for biases in stellar effective temperatures derived by different methods.  For example, the APOGEE spectrograph ($1.5-1.70 \;\mu$m at $R=22,500$) measured effective temperatures for 3493 young stars finding offsets in $\teff$ of typically 200$-$500 K and as high as 1000 K compared to previous studies \citep{2014ApJ...794..125C}.  Low resolution optical spectroscopy shows a typical spread of 200 K in the spectral type-to-effective-temperature conversion scale \citep{2014ApJ...786...97H}.  Stellar evolution models including the effect of starspots can make a coeval 10 Myr population exhibit apparent age spreads of 3$-$10 Myr, with derived masses biased towards lower masses.  The observations of sunspots are lagging behind the theory \citep{2015ApJ...807..174S}.

The importance of starspots needs to be empirically evaluated to assess systematic biases of pre-main sequence stellar evolutionary models.

In this paper we characterize the starspot coverage and temperature contrast of \name using full spectrum fitting.  We chose \name for its likelihood of large areal coverage of starspots, as evinced by large-amplitude photometric variability  detected in $BVRI$ bands.  Its period is 3.37$\pm$0.01 days \citep{1993AJ....106.1608V,1994IBVS.4042....1G}.  Recent optical spectro-polarimetry showed evidence for hot or cool starspots covering an estimated $25\%$ of the stellar surface \citep{2014MNRAS.444.3220D}.  The source has little or no accretion ($\log{\dot M} < 8.1$), no detection of mid-infrared nor mm excess, and no stellar companion, so its spectrum should be devoid of complicating factors like near-IR excess veiling and accretion excess.  This source is the ideal candidate for direct measurement of two-temperature photosphere due to starspots.

We extend a new spectral forward modeling framework to include two-temperature photosphere models.  We demonstrate the constraining power of full spectrum fitting with panchromatic optical and near-IR echelle spectroscopy.  We show that relatively small starspot filling factors can be detected due to the large spectral grasp of IGRINS.  

\section{Methodology}\label{sec:methods} 



\section{Observations}\label{sec:obs} 

\subsection{IGRINS Spectroscopy}\label{sec:igrins} 
We acquired observations with IGRINS on the Harlan J. Smith Telescope at McDonald Observatory on 2015-11-18 $09^h$ UTC.  The Immersion Grating Infrared Spectrograph, IGRINS \citep{2014SPIE.9147E..1DP,2012SPIE.8450E..2SG}, is a high resolution near-infrared echelle spectrograph providing simultaneous $R\simeq45,000$ spectra over 1.48-2.48\um.  The spectrograph has two arms with 28 orders in $H-$band and 25 orders in $K-band$.

We combined our IGRINS spectroscopy with existing ESPaDOnS panchromatic optical echelle spectra to provide 88 spectral orders between 5100 and 25000 \AA \citep{2014MNRAS.444.3220D}.

\section{Analysis}
We compared the IGRINS spectrum of \name to the \PHOENIX grid of pre-computed synthetic model spectra \citep{2013A&A...553A...6H}.  

We use the \texttt{Starfish} code \citep{2015ApJ...812..128C}.  In this Section we describe our modifications to to the code.

The existing \texttt{Starfish} implementation \emph{simultaneously} fits both the intrinsic stellar parameters, $\vt_{\ast} = \{\teff, \logg, \Z \}$, and extrinsic stellar parameters, $\vt_{\rm ext} = \{\sigma_v, \vsini, v_r, \Omega, A_V\}$, with all $N_{\mathrm{ord}}$ spectral orders sharing the same stellar parameters\footnote{The implementation \emph{also} simultaneously fits a small-amplitude polynomial to each spectral order to adjust for slight imperferfections in the continuum shape left over from the blaze correction in the IGRINS pipeline.  The coefficients of the polynomials are treated as nuisance parameters.  This strategy makes us resilient against the pitfalls of pseudo-continuum-fitting, and effortlessly propagates the uncertainty attributable to the continuum level into the uncertainty in the astrophysically interesting stellar parameters.}.  We modified the default implementation by chunking the spectrum into 58 of its cleanest spectral orders and deriving the stellar parameters $\vt_{\ast,m}$ \emph{independently} in each $m^{\mathrm{th}}$ order.  We therefore have 58 unique estimates for $\teff$ as a function of wavelength, which we compile in Figure \ref{fig:teffOrder}.  The figure shows a dramatic effect: \textbf{The derived effective temperature is on-average about 800 K cooler in the $K-$band than in the optical.}  

\section{Results}
We interpret the cooler temperatures derived at $K-$band as the presence of sunspots contributing more flux at longer wavelengths, as shown in the bottom panel of Figure \ref{fig:teffOrder}.  Here we show the absolute flux ratio, $f_{\lambda, B} / f_{\lambda, A}$ at two different temperatures: $T_\textrm{eff,A} = 4100$ and $T_\textrm{eff,B} = 3300$.  We present the ratio of two synthetic spectra from Phoenix (blue solid line), and the black body ratio (red dashed line).  

We further performed a plausibility check to constrain the effect size of starspots.  We generated flux calibrated spectra at two temperatures, and forward-modeled them to resemble the LkCa4 IGRINS and ESPaDOnS spectra in all other ways (\emph{i.e.} $\logg, \Z, \vt_{\rm ext}$), including noise and calibration parameters.  We coadded the spectra in a mixture model:  $ \mathsf{M}_{mix} = c \cdot \mathsf{M}_A(T_\textrm{eff,A}) + (1-c) \cdot \mathsf{M}_B(T_\textrm{eff,B})$.  We chose a fill factor of starspots of 30\%.  We then re-ran our single-temperature fitting procedure on the two-temperature, synthetic, noised-up data to see what stellar parameters one would na\"{\i}vely derive.  The noised-up optical data yields $\teff \sim 4100$ K, close to that of the A component.  We are awaiting the results from $K-$ band at the time of writing.


\subsection{Deriving $\teff$ in each IGRINS order}

We want to measure the derived effective temperature as a function of wavelength.  The motivation for this strategy is as follows.  We assume that LkCa4 has a significant fraction of its stellar disk covered by starspots citeXX.  Such a photosphere will have an emergent spectrum composed of superpositions of relatively cool patches and warm patches.  The visible portion of the spectrum will be dominated by the warm patches, but the contrast with the starspots will be higher.  The infrared spectrum will have a lower overall spot contast, but with higher relative contribution attributable to the cool patches citeXX.  

The fit of a single-temperature forward-modeled spectrum to different regions of the spectrum could therefore conceivably yield different estimates for the effective temperature.

In principle, one could estimate the effective temperature \emph{evaluated independently for each spectral line}.  The line strength is often degenerate with other properties (\emph{e.g.} magnetic field, $\log{g}$, $[\mathrm{Fe}/\mathrm{H}]$, etc.), but \emph{on average} these correlations would be averaged over.

We measured the effective temperature in \emph{each of the 43 spectral orders} of IGRINS.  Figure \ref{fig:teffOrder} plots each derived effective temperature value and uncertainty at the center position of each order.

\begin{figure*}
	\centering
	\includegraphics[width=0.95\textwidth]{figures/teff_v_wl_example} 
	\caption{Effective temperature as derived by different orders.}
	\label{fig:teffOrder}
\end{figure*}


\section{Spectral properties}\label{sec:lines}


\appendix

\section{Technical details}

Here are some specifics about what we did.

\section{Previous work}

Table REF lists measurements of LkCa4 from previous studies.
%%%%%%%%%%%%%%%%%%%%%%%%%%%%%%%%%%%%%%%%
% TABLE - History of LkCa4
%%%%%%%%%%%%%%%%%%%%%%%%%%%%%%%%%%%%%%%%
%\begin{deluxetable*}{lccccccccc}
\begin{deluxetable}{p{4cm}ccccccccc}

\tabcolsep=0.11cm
%\rotate
\tabletypesize{\footnotesize}
\tablecaption{Previous studies of LkCa4\label{tbl_history}}
\tablewidth{0pt}
\tablehead{
\colhead{Ref} &
\colhead{Band(s)} &
\colhead{Resolution} &
\colhead{Classification} &
\colhead{$T_{eff}$} &
\colhead{$\log{g}$} &
\colhead{$A_V$} &
\colhead{[Fe/H]} &
\colhead{$v\sin{i}$} &
\colhead{$v_{z}$} \\
\colhead{} &
\colhead{} &
\colhead{} &
\colhead{} &
\colhead{K} &
\colhead{} &
\colhead{} &
\colhead{km/s} &
\colhead{km/s} &
}
\startdata
 Junk et al. & $V$ & 10000 & - &4000 & 3.9 & 0.0 & 0.0 & 20.1 & 15 \\
 \citet{1986AJ.....91..575H} & $V$ &  & K7 V & - & - & - & - & 26.1$\pm$2.4 & +13$\pm$4 \\
 \citet{1987AJ.....93..907H} & $U$ & $<$2 km/s & T-Tauri & - & - & - & - & 26.1$\pm$2.4 & +16.9$\pm$2.6 \\
 \citet{1988AJ.....96..777D} & $V$ & 13$\AA$ & Me & - & - & - & - & - & - \\
 \citet{1989AJ.....97.1451S} & $V$ & - & K7:V & - & - & 0.95 & - & - & - \\
 \citet{1989AJ.....98.1444S} & $V$ & 0.3$\AA$ & K7:V & - & - & - & - & - & - \\
 \citet{1994ApJ...424..237S} & $V$ & - & K7 & 4000 & - & 1.25 & - & - & - \\
% \citet{1994A&A...282..503M} & $V$ & ? & K7 & 4130 & 3.65 & - & - & - & - \\	
 \citet{1995ApJS..101..117K} & ? & ? & K7 & 4060 & 3.65 & 0.69 & - & - & - \\
 \citet{1995ApJ...452..736H} & ? & ? & K7 & 4000 & - & 0.68 & - & - & - \\
\enddata

\tablecomments{Some values are not original, see references to trace to original source.}
%\tablerefs{}

%\end{deluxetable*}
\end{deluxetable}

In short, here is what we know about \name.  It is a weak-lined \emph{T-Tauri} star.  It exhibits a 3.36-3.37 day period \citep{1993AJ....106.1608V,1994IBVS.4042....1G}, as evinced by photometric monitoring in $BVRI$ bands.  No periodic signal is detected at $U-$band.  Section 5 of \citet{1993AJ....106.1608V}, ``More on the nature of CTT and WTT spots'' goes into detail about the whether the photometric variability is attributable to hot or cold patches.  \name has no detected M-type-or-earlier binary companion down to $0.13''$ \citep{1993A&A...278..129L}.  \name demonstrates variability in X-rays with ROSAT \citep{1994ApJ...424..237S}, with $L_{x}=2.4$ ergs/s.


\acknowledgements
The authors thank Gregory N. Mace for carrying out the IGRINS observations. This research has made use of NASA's Astrophysics Data System.

{\it Facilities:} \facility{Smith (IGRINS)}

\clearpage

\bibliographystyle{apj}
\bibliography{ms}

\end{document}


