%\documentclass[iop,revtex4]{emulateapj}% change onecolumn to iop for fancy, iop to twocolumn for manuscript
\documentclass[onecolumn]{emulateapj}% change onecolumn to iop for fancy, iop to onecolumn for manuscript
%\documentclass[12pt,preprint]{aastex}

%\usepackage{lineno}
%\usepackage{blindtext}
%\linenumbers

\let\pwiflocal=\iffalse \let\pwifjournal=\iffalse
%From: http://arxiv.org/format/1512.00483
%\input{setup}
\usepackage{enumerate}
\usepackage{amsmath,amssymb}
\usepackage{bm}
\usepackage{color}
\usepackage[utf8]{inputenc}

%% For anyone who downloaded my source file from arxiv:
%% I stole most of this setup.tex from a paper by Peter .K.G. Williams, but I made a bunch of edits to satisfy my own needs. You might check his paper out (http://arxiv.org/abs/1409.4411) for the original source file or contact him if you have any questions, since I don't really understand how some of these things work. 
%One cool thing it does is you can define an object, just that when someone clicks on the pdf it will link to simbad. I could never quite get this to work, probably because you have to get the text exactly right and my motivation for getting it to work was not super high. 


% basic packages
\usepackage{amsmath,amssymb}
\usepackage[breaklinks,colorlinks,urlcolor=blue,citecolor=blue,linkcolor=blue]{hyperref}
\usepackage{epsfig}    
\usepackage{graphicx}    
\usepackage{lineno}
\usepackage{natbib}
\usepackage{bigints}
\usepackage[outdir=./]{epstopdf}



% font stuff
\usepackage[T1]{fontenc}
\pwifjournal\else
  \usepackage{microtype}
\fi


% emulateapj has overly conservative figure widths, I think because some
% people's figures don't have good margins. Override.
\pwifjournal\else
  \makeatletter
  \renewcommand\plotone[1]{%
    \centering \leavevmode \setlength{\plot@width}{0.95\linewidth}
    \includegraphics[width={\eps@scaling\plot@width}]{#1}%
  }%
  \makeatother
\fi


\makeatletter

\newcommand\@simpfx{http://simbad.u-strasbg.fr/simbad/sim-id?Ident=}

\newcommand\MakeObj[4][\@empty]{% [shortname]{ident}{url-escaped}{formalname}
  \pwifjournal%
    \expandafter\newcommand\csname pkgwobj@c@#2\endcsname[1]{\protect\object[#4]{##1}}%
  \else%
    \expandafter\newcommand\csname pkgwobj@c@#2\endcsname[1]{\href{\@simpfx #3}{##1}}%
  \fi%
  \expandafter\newcommand\csname pkgwobj@f#2\endcsname{#4}%
  \ifx\@empty#1%
    \expandafter\newcommand\csname pkgwobj@s#2\endcsname{#4}%
  \else%
    \expandafter\newcommand\csname pkgwobj@s#2\endcsname{#1}%
  \fi}%

\newcommand\MakeTrunc[2]{% {ident}{truncname}
  \expandafter\newcommand\csname pkgwobj@t#1\endcsname{#2}}%

\newcommand{\obj}[1]{%
  \expandafter\ifx\csname pkgwobj@c@#1\endcsname\relax%
    \textbf{[unknown object!]}%
  \else%
    \csname pkgwobj@c@#1\endcsname{\csname pkgwobj@s#1\endcsname}%
  \fi}
\newcommand{\objf}[1]{%
  \expandafter\ifx\csname pkgwobj@c@#1\endcsname\relax%
    \textbf{[unknown object!]}%
  \else%
    \csname pkgwobj@c@#1\endcsname{\csname pkgwobj@f#1\endcsname}%
  \fi}
\newcommand{\objt}[1]{%
  \expandafter\ifx\csname pkgwobj@c@#1\endcsname\relax%
    \textbf{[unknown object!]}%
  \else%
    \csname pkgwobj@c@#1\endcsname{\csname pkgwobj@t#1\endcsname}%
  \fi}

\makeatother


% Evil magic to patch natbib to only highlight the year paper refs, not the
% authors too; as seen in ApJ. From
% http://tex.stackexchange.com/questions/23227/.

\pwifjournal\else
  \usepackage{etoolbox}
  \makeatletter
  \patchcmd{\NAT@citex}
    {\@citea\NAT@hyper@{%
       \NAT@nmfmt{\NAT@nm}%
       \hyper@natlinkbreak{\NAT@aysep\NAT@spacechar}{\@citeb\@extra@b@citeb}%
       \NAT@date}}
    {\@citea\NAT@nmfmt{\NAT@nm}%
     \NAT@aysep\NAT@spacechar\NAT@hyper@{\NAT@date}}{}{}
  \patchcmd{\NAT@citex}
    {\@citea\NAT@hyper@{%
       \NAT@nmfmt{\NAT@nm}%
       \hyper@natlinkbreak{\NAT@spacechar\NAT@@open\if*#1*\else#1\NAT@spacechar\fi}%
         {\@citeb\@extra@b@citeb}%
       \NAT@date}}
    {\@citea\NAT@nmfmt{\NAT@nm}%
     \NAT@spacechar\NAT@@open\if*#1*\else#1\NAT@spacechar\fi\NAT@hyper@{\NAT@date}}
    {}{}
  \makeatother
\fi

\newcommand{\prob}{{\rm prob}}
\newcommand{\qN}{\{q_i\}_{i=1}^N}
\newcommand{\qM}{\{q_{im}\}_{i=1,m=0}^{N,M}}
\newcommand{\yN}{\{y_i\}_{i=1}^N}

\newcommand{\kms}{ \textrm{km s}^{-1} }

\newcommand{\vM}{\mathsf{M}}
\newcommand{\vD}{\mathsf{D}}
\newcommand{\vR}{\mathsf{R}}
\newcommand{\vC}{\mathsf{C}}
\newcommand{\fM}{ \vec{{\bm M}}}
\newcommand{\fMi}{M_i}
\newcommand{\fD}{ \vec{{\bm D}}}
\newcommand{\fDi}{D_i}
\newcommand{\fR}{ {\bm R}}
\newcommand{\dd}{\,{\rm d}}
\newcommand{\trans}{\mathsf{T}}
\newcommand{\teff}{T_\textrm{eff}}
\newcommand{\teffa}{T_\textrm{amb}}
\newcommand{\teffb}{T_\textrm{spot}}
\newcommand{\logg}{\log g}
\newcommand{\Z}{[{\rm Fe}/{\rm H}]}
\newcommand{\A}{[\alpha/{\rm Fe}]}
\newcommand{\vsini}{v \sin i}
\newcommand{\matern}{Mat\'{e}rn}
\newcommand{\HK}{$\textrm{H}_2$O-K2}
\newcommand{\cc}[2]{c_{#2}^{(#1)}} 

\newcommand{\flam}{f_\lambda}
\newcommand{\vt}{ {\bm \theta}}
\newcommand{\vT}{ {\bm \Theta}}
\newcommand{\vp}{ {\bm \phi}}
\newcommand{\vP}{ {\bm \Phi}}
\newcommand{\cheb}{ \vp_{\mathsf{P}}}
\newcommand{\chebi}[1]{ \vp_{\textrm{Cheb}_{#1}}}
\newcommand{\Cheb}{ \vP_{\textrm{Cheb}}}
\newcommand{\Chebi}[1]{ \vP_{\textrm{Cheb}_{\ne #1}}} 
\newcommand{\cov}{ \vp_{\mathsf{C}}}
\newcommand{\covi}[1]{ \vp_{\textrm{cov}_{#1}}} 
\newcommand{\Cov}{ \vP_{\textrm{cov}}}
\newcommand{\Covi}[1]{ \vP_{\textrm{cov}_{\ne #1}}} 

\newcommand{\allParameters}{\vT} 
\newcommand{\nuisanceParameters}{\vP} 

\newcommand{\KK}{\mathcal{K}}
\newcommand{\Kglobal}{\KK^{\textrm{G}}}
\newcommand{\Klocal}{\KK^{\textrm{L}}}

\newcommand{\Gl}{Gl\,51}
\newcommand{\PHOENIX}{{\sc Phoenix}}

% Appendix commands
\newcommand{\wg}{\mathbf{w}^\textrm{grid}}
\newcommand{\wgh}{\hat{\mathbf{w}}^\textrm{grid}}

\newcommand{\Sg}{\mathbf{\Sigma}^\textrm{grid}}


\newcommand{\todo}[1]{ \textcolor{blue}{\\TODO: #1}}
\newcommand{\comm}[1]{ \textcolor{red}{MGS: #1}}
\newcommand{\hili}[1]{ \textcolor{green}{#1}}
\newcommand{\ctext}[1]{ \textcolor{blue}{\% #1}}


%  From Peter Williams and Andy Mann again:
\newcommand{\um}{$\mu$m}


\newcommand{\iancze}{{\sc C15 }}

\providecommand{\eprint}[1]{\href{http://arxiv.org/abs/#1}{#1}}
\providecommand{\adsurl}[1]{\href{#1}{ADS}}
\newcommand{\name}{LkCa4 }
%\def\vsini{$v\sin{i_*}$}

\slugcomment{In preparation}

\shorttitle{\name IGRINS spectroscopy}

\shortauthors{Gully-Santiago et al.}

\bibliographystyle{yahapj}

\begin{document}
 
\title{IGRINS spectra of \name}

\author{Michael A. Gully-Santiago,\altaffilmark{1} Greg Herczeg,\altaffilmark{1} et al.}


\altaffiltext{1}{Kavli Institute for Astronomy and Astrophysics, Beijing, China}

\begin{abstract}
We interpret the high resolution near-IR spectra of \name.  We used the instrument IGRINS.
\end{abstract}

\keywords{stars: fundamental parameters --- stars: individual (\name) ---  stars: low-mass -- stars: statistics}

\maketitle

\section{Introduction}\label{sec:intro}

\subsection{Pre Main Sequence HR diagram spread}

For a single presumably coeval stellar cluster, pre main-sequence HR diagrams exhibit large ($\delta t_{\ast} \sim t_{\ast}$) apparent spreads in age \citep[e.g.][]{2011A&A...534A..83R}. The large spreads in these pre main-sequence HR diagrams is controversial: some take it as evidence for intrinsic age spreads, some take it as evidence for non static accretion history \citep{2009ApJ...702L..27B, 2010ARA&A..48..581S}.  

Still others take the large apparent age spreads in pre main-sequence HR diagrams as evidence for the limited physics included in the pre main-sequence evolutionary model isochrones.  Most modern evolutionary models do not yet include the effects of starspots or magnetic fields, despite observations betraying their presence, \emph{e.g.} optical monitoring and Zeeman doppler imaging \citep{2008A&A...479..827G,2014MNRAS.444.3220D}.  

\subsection{Previous studies of Starspots}

Meanwhile, sunspots are probably responsible for biases in stellar effective temperatures derived by different methods.  For example, the APOGEE spectrograph ($1.5-1.70 \;\mu$m at $R=22,500$) measured effective temperatures for 3493 young stars finding offsets in $\teff$ of typically 200$-$500 K and as high as 1000 K compared to previous studies \citep{2014ApJ...794..125C}.  Low resolution optical spectroscopy shows a typical spread of 200 K in the spectral type-to-effective-temperature conversion scale \citep{2014ApJ...786...97H}.  Stellar evolution models including the effect of starspots can make a coeval 10 Myr population exhibit apparent age spreads of 3$-$10 Myr, with derived masses biased towards lower masses.  The observations of sunspots are lagging behind the theory \citep{2015ApJ...807..174S}.

The importance of starspots needs to be empirically evaluated to assess systematic biases of pre-main sequence stellar evolutionary models.

\subsection{Previous studies of \name}

In this paper we characterize the starspot coverage and temperature contrast of \name using full spectrum fitting.  We chose \name for its likelihood of large areal coverage of starspots, as evinced by large-amplitude photometric variability  detected in $BVRI$ bands.  Its period is 3.37$\pm$0.01 days \citep{1993AJ....106.1608V,1994IBVS.4042....1G}.  Recent optical spectro-polarimetry showed evidence for hot or cool starspots covering an estimated $25\%$ of the stellar surface \citep{2014MNRAS.444.3220D}.  The source has little or no accretion ($\log{\dot M} < 8.1$), no detection of mid-infrared nor mm excess, and no stellar companion, so its spectrum should be devoid of complicating factors like near-IR excess veiling and accretion excess.  This source is the ideal candidate for direct measurement of two-temperature photosphere due to starspots.

We extend a new spectral forward modeling framework to include two-temperature photosphere models.  We demonstrate the constraining power of full spectrum fitting with panchromatic optical and near-IR echelle spectroscopy.  We show that relatively small starspot filling factors can be detected due to the large spectral grasp of IGRINS.  

\section{Methodology}\label{sec:methods} 


\subsection{Full-spectrum fitting framework}
% Motivation: The questions we want to answer
Modeling the periodic light curves of spotted stars can answer these questions: ``What evidence is there to support the existence of starspots.  What constraint can be placed on the effective temperature contrast and areal coverage of the starspots?  What is the longitudinal assymetry of the spots?  ''.  As pointed out by CITE XX, cyclical light curve variations indicate only the longitudinally assymetric component of the star spots.  Photometric modeling of the light curve produces degenerate estimates in the temperature contrast and areal coverage fraction.  For example, CITE XX achieved XX.  

By modeling the composite \emph{spectrum} of a spotted star, it is possible to directly measure the collective areal coverage and effective temperature contrast.  A stellar spectrum with a longitudinaly symmetric distribution of starspots, or a pole-on star with a single spot, or a non-rotating star would all show zero amplitude of photometric variability, yet they all contain star spots.  Spectroscopic modeling can still measure the starspot properties in these cases.  

In principle, more advanced questions could be asked about the properties of the spots: ``How many distinct spot groups are there?  Is there evidence for both hot spots and cool spots?  What is the longitudinal distribution of the spots?  What is the (co-)latitudinal distribution of the spots?  What is the distribution of effective temperature within a spot or spot group?  What are the morphologies of the spots?''.  Solving these questions would require a complex model comparison framework, combining all available data and lines of evidence from photometric, spectroscopic, and spectropolarimetric monitoring.  We seek to answer the much easier questions about the aggregate properties of the stellar photosphere, assuming the star is well described by two distinct photospheric components.  

% Motivation:  Why sampling?  Why full spectrum fitting?

A main motivating factor in designing our methodology was our wish to adress this question for a given spectrum: ``Is the spectrum consistent with the non-detection of starspots, and to what level of confidence can the detection of starspots be reported?''.  In order to answer this question, we need some way to gauge the uncertainty in the derived stellar properties, which we assumed would have some level of degeneracy among themselves.  For example, we expect the best fit effective temperature contrast will be partially degenerate with the areal coverage fraction: a slightly cooler, smaller sunspot can masquerade as a larger, warmer spot.  This degeneracy should be lessened in spectroscopic methods compared to photometric methods.  The desire to reveal these intra-parameter degeneracies motivated the choice of \emph{sampling} from the posterior probability density distribution function.

% Overview and summary of existing model

We adopted and extended an existing modular framework for spectral inference \citep[][hereafter \iancze]{2015ApJ...812..128C}.  We briefly summarize the model here for carity of notation, but see \iancze for more details.  We follow and expand the notation from \iancze, all summarized in Table \ref{table:params}. 

The main idea is that \iancze define a pixel-level forward model $\vM$ of the data.  A flexible Gaussian Process noise model handles correlations in the residual spectrum, $\vR= \vD - \vM$.  The model encapsulates the major intrinstic stellar parameters $\vt_{\ast}$ and the extrinsic stellar parameters $\vt_{\rm ext}$:

\begin{eqnarray} \label{eqn:scaling}
\vM(\vT) &=& \vM(\vt_{\ast}, \vt_{\rm ext}) \\
         &=& \vM(\vt_{\ast}, \sigma_v, v\sin{i}, v_r) \times \Omega \times 10^{-0.4\,A_{\lambda}}, \nonumber
\end{eqnarray}

The $\vt_{\ast}$ are comprised of the dimensions of the pre-computed stellar model grid.  We use the \PHOENIX model grid \citep{2013A&A...553A...6H}, with fixed solar alpha abundances: $\A = 0$, so $\vt_{\ast} = \{\teff, \logg, \Z \}$.

\subsection{Approach 1: Deriving unique Teff in each order}

The model $\vM(\vT)$ was intended to apply a single set of stellar parameters to the entire wavelength range of interest, presumably the entire available spectrum.  For a spectrum from a multi-order echelle spectrograph, this means each spectral order is fit with the same stellar parameters.  Simultaneous fitting is the desired behavior for most applications involving a single stellar photospheric component.  In our case, however, we know that a single stellar photospheric component is a poor model for the data: there are \emph{two} components lurking in the composite spectrum.  But depending on the star spot properties, a single stellar photosphere component might fit the composite spectrum satisfactorily, like if the flux contrast of the starspot is simply too low compared to the noise.

The starspot contrast depends on wavelength, roughly as the ratio of blackbodies.  Figure XX shows a plot of the flux ratio of two stellar photospheres with identical stellar properties except for different effective temperatures.  So a sufficiently large bandwidth spectrum could exhibit differing levels of fit quality as a function of wavelength, if a single temperature photosphere fit is naively applied to different chunks of wavelengths.  

We modified the \iancze spectral inference framework to fit $\vM$ in each spectral order \emph{independently}: $\vT \rightarrow \vT_o$, where $o$ signifies the spectral order, which could stem from multiple spectrographs, each with multiple orders.  This change does not represent reality: the star has a single set of stellar properties, which do not depend on the wavelength range at which they are measured.  The na\"ive single-component fit to different wavelength provide a plausibility argument for the presence of a second photospheric component if the derived $\vT_o$ show a spread in $\teff$.  It is conceivable that other parameters, like $\logg$ could show a dependence on wavelength if non-standard physics---\emph{i.e.}physics not included in the pre-computed model grid---mimics the effect of \emph{e.g.} $\logg$ on the spectrum.  For example, magnetic fields could cause Zeeman broadening, which could alter the derived stellar parameters $\vT_o$.  For these reasons, we intentionally allowed \emph{all} the stellar parameters $\vT_o = \{T_{\mathrm{eff},o}, \log{g_o}, \Z_o, \Omega_{o}, v\sin{i}_o, v_{r,o}\}$ to vary by order.  

In principle, one could refine the inference on $T_{\mathrm{eff},o}$ in an \emph{empirical Bayes'} strategy CITE (XX, astroML book?), in which we re-evaluate $\vT_o$ by setting priors on all the stellar parameters except $T_{\mathrm{eff},o}$.  We elected not to pursue an empirical Bayes' approach.

As a side benefit, the strategy of fitting $\vT_o$ in each spectral order provides a useful check on our instrumental calibration. If the stellar parameters in one order are routinely discrepant, we can examine and refine our instrumental calibration for that spectral order.  For example, the radial velocity $v_r$ could be systematically offset in one order due to instrumental calibration problems.  Or perhaps the $v\sin{i}$ could show a dependence on wavelength, since the spectral inference framework from \iancze assumes the spectral resolution is fixed across all wavelengths, but typical spectrographs exhibit a small dependence of spectral resolution on wavelength.  

Spectral-order level granularity in fit quality is not available in a fit that includes all $N_{order}$ spectral orders.  In fact, the benefits of spectral chunking could be made \emph{even more granular} by chunking the spectrum into individual spectral lines or line-groups.  The fit quality could then be assessed on a line-by-line basis, even taking into account systematic effects like individual elemental abundance patterns beyond the coarse $\Z$ estimates.  Future improvements to the spectral inference framework are directed at implementing this spectral-line chunking strategy.  


\subsection{Approach 2: Two component photosphere mixture model}

We construct a mixture model spectrum of the form:

\begin{eqnarray} \label{eqn:mix_M}
\vM_{\mathrm{mix}}(\vT) = \vM(\teffa, ...) \times \Omega_a + \vM(\teffb, ...) \times \Omega_b
\end{eqnarray}


Where $\teffa, \teffb, \Omega_a, \Omega_b$ are temperatures and solid angles of two photospheric components respectively.  All other stellar intrinsic and extrinsic parameters, and instrumental nuisance calibration parameters are shared among the two components, as chronicled in Table \ref{table:params}.

If the stellar models are in absolute flux units, our mixture model is complete as stated in Equation \ref{eqn:mix_M}.  However \iancze employ \emph{standardized} $\flam$ when constructing the forward model $\vM$:


\begin{eqnarray} \label{eqn:normalization}
\bar \flam = \frac{\flam}{\int_{0}^{\infty} \flam d\lambda} = \frac{\flam}{f}
\end{eqnarray}

The reason for normalizing is a matter of practicality: the number of PCA eigenspectra components in the spectral emulator scales steeply with the pixel-to-pixel variance of $f_{\lambda}(\{\vt_{\ast}\}^\textrm{grid})$, increasing the computational cost of spectral emulation.  The choice to standardize fluxes makes no difference for modeling a single photospheric component.  But for two-component photosphere models, the relative flux of the two model spectra needs to be accounted for to get an accurate estimate of the areal coverage fraction of the cool spots, $c \equiv \Omega_b/(\Omega_a+\Omega_b)$.  So we scale the mixture model in the following way:

\begin{eqnarray} \label{eqn:norm_scaling}
f_{\lambda, \mathrm{mix}} &=& f_{\mathrm{a}} \bar f_{\lambda, \mathrm{a}} \times \Omega_a + f_{\mathrm{b}} \bar f_{\lambda, \mathrm{b}} \times \Omega_b \\
q &=& q(\vt_{\ast})\equiv \frac{f_{\mathrm{b}}(\teffb, ...)}{f_{\mathrm{a}}(\teffa, ...)} \\
f_{\lambda, \mathrm{mix}}^{\prime} &=& \bar f_{\lambda, \mathrm{a}} \times \Omega_a + q \bar f_{\lambda, \mathrm{b}} \times \Omega_b
\end{eqnarray}

where the prime symbol in the final line indicates a re-standarized mixture model flux, where the relative fluxes of the model components are now correctly scaled.

We are then tasked with computing an estimator, $\hat f(\vt_{\ast})$, for estimating the scale factor $q$ in-between model gridpoints.  The Right Thing to Do would be to follow \iancze by training a Gaussian process on the $f(\{\vt_{\ast}\}^\textrm{grid})$.  What We Actually Did was linearly interpolate between the model grid-points.  Interpolation can cause pile-up near model grid-points, as noted in \citet{2014ApJ...794..125C}, which motivated the spectral emulation procedure in \iancze.  We assume the interpolation of $\hat f$ is smooth enough that we will not see such pileups and if even we did see pileups, they would mostly be discernable in the distribution of samples in the starspot areal coverage fraction $c$.  One drawback of our estimator method compared to the Gaussian Process regression method is that we do not propagate the uncertainty associated with the absolute flux ratio interpolation into our estimate of $c$.  We assume this uncertainty is relatively small and can be ignored.

The mixture model is a linear operation with all the same stellar extrinsic parameters, so we can re-use all the same post-processed eigenspectra $\widetilde{\mathbf{\Xi}}$, mean spectrum $\widetilde{\xi}_\mu$, and variance spectrum $\widetilde{\xi}_\sigma$, with the tildes representing all post processing \emph{except} the $\Omega$ scaling.  We calulate \emph{two} sets of eigenspectra weights $\mathbf{w}_{\in (\mathrm{a}, \mathrm{b})}$, and their associated mean and covariances following the Appendix of \iancze, and yielding:

\begin{equation}
  \mathsf{M}_{\mathrm{mix}}^\prime = \Omega_a (\widetilde{\xi}_\mu + \mathbf{X} \mathbf{\mu}_{\mathbf{w}, \mathrm{a}}) + q \Omega_b (\widetilde{\xi}_\mu + \mathbf{X} \mathbf{\mu}_{\mathbf{w}, \mathrm{b}})
\end{equation}

\begin{equation}
  \mathsf{C}_{\mathrm{mix}}^\prime = \Omega_a^2 \mathbf{X} \mathbf{\Sigma}_\mathbf{w, \mathrm{a}} \mathbf{X}^T + q \Omega_b^2 \mathbf{X} \mathbf{\Sigma}_\mathbf{w, \mathrm{b}} \mathbf{X}^T + \mathsf{C}
  \label{eqn:modC}
\end{equation}



\begin{deluxetable}{clcc}[!bh]
\tablecaption{\label{table:params} Definition of symbols and parameters used}
\tablehead{
\colhead{Label} &
\colhead{Description} & 
\colhead{Shared?} &
\colhead{Source}
}
\startdata
$\vM$ & Pixel-level model & - & \iancze \\
$\vT$ & Stellar parameters in the model & - & \iancze \\
$q_o(\vt_{\ast})$ & Absolute mean flux ratio in order $o$  & N & This Work \\
\hline
 \multicolumn{4}{c}{$\vt_{\ast}$} \\
\hline
$\teffa$ & Effective temperature of component A & N & This Work \\
$\teffb$ & Effective temperature of component B & N & This Work \\
$\Delta \teff$ & Spot temperature contrast $\teffa - \teffb $ & - & This Work \\
$\logg$ & Stellar surface gravity & Y & \iancze \\
$\Z$ & Metallicity & Y & \iancze \\
\hline
\multicolumn{4}{c}{$\vt_{\rm ext}$} \\
\hline
$v\sin{i}$ & Projected stellar rotation & Y$^a$ & \iancze \\
$\sigma_v$ & Instrumental resolution & Y & \iancze \\
$\Omega_a$ & Solid angle subtended by component A & N & This Work \\
$\Omega_b$ & Solid angle subtended by component B & N & This Work \\
$c$ & Spot fill factor ($\Omega_b / \Omega_a$) & N & This Work \\
$A_{\lambda}$ & Extinction towards the source 2 & Y & \iancze \\
\enddata
\tablecomments{\emph{a}- In principle, star spots could have non-uniform latitudinal distributions, resulting in slight differences in $v\sin{i}$ between the components CITE XX-McDonald Obs.  We ignore this effect for now, but it could be an interesting line of study for future work.}
\end{deluxetable}


\subsection{Challenges: Tuning the transition probability matrix}

Once we had the mixture model in hand, we could apply the blocked Gibbs MCMC sampling framework from \iancze.  We have added 2 more parameters ($c, \teffb$) to the model, expanding from 6 to 8 total stellar parameters.  Unfortunately this increase in dimensionality adds considerable complexity to tuning the model.  The number of possible tuning parameters in a conventional Metropolis Hastings MCMC sampler scales as $N(N+1)/2$, so our 2 new parameters increased the tuning parameters from 15 to 36!  The tuning parameters are covariances between dimensions in your proposal distribution.  Luckily, most of the covariances are negligibly small.  However the $c, \teffb, \Omega$ terms were all highly correlated, suggesting an affine transformation of the parameters would accelerate the convergence of the MCMC chains.


\section{Observations}\label{sec:obs} 

\subsection{IGRINS Spectroscopy}\label{sec:igrins} 
We acquired observations with IGRINS on the Harlan J. Smith Telescope at McDonald Observatory on 2015-11-18 $09^h$ UTC.  The Immersion Grating Infrared Spectrograph, IGRINS \citep{2014SPIE.9147E..1DP,2012SPIE.8450E..2SG}, is a high resolution near-infrared echelle spectrograph providing simultaneous $R\simeq45,000$ spectra over 1.48-2.48\um.  The spectrograph has two arms with 28 orders in $H-$band and 25 orders in $K-band$.

\subsection{ESPaDOnS Spectroscopy}
We combined our IGRINS spectroscopy with existing ESPaDOnS panchromatic optical echelle spectra to provide 88 spectral orders between 5100 and 25000 \AA \citep{2014MNRAS.444.3220D}.

\subsection{At what phase of variability were the spectra acquired?}


\section{Results}

\subsection{Result 1: Signal from deriving $\teff$ in each IGRINS order}

We compared the IGRINS spectrum of \name to the \PHOENIX grid of pre-computed synthetic model spectra \citep{2013A&A...553A...6H}.  

We modified the default implementation by chunking the spectrum into 58 of its cleanest spectral orders and deriving the stellar parameters $\vt_{\ast,m}$ \emph{independently} in each $m^{\mathrm{th}}$ order.  We therefore have 58 unique estimates for $\teff$ as a function of wavelength, which we compile in Figure \ref{fig:teffOrder}.  The figure shows a dramatic effect: \textbf{The derived effective temperature is on-average about 800 K cooler in the $K-$band than in the optical.}  

We interpret the cooler temperatures derived at $K-$band as the presence of sunspots contributing more flux at longer wavelengths, as shown in the bottom panel of Figure \ref{fig:teffOrder}.  Here we show the absolute flux ratio, $f_{\lambda, B} / f_{\lambda, A}$ at two different temperatures: $T_\textrm{eff,A} = 4100$ and $T_\textrm{eff,B} = 3300$.  We present the ratio of two synthetic spectra from Phoenix (blue solid line), and the black body ratio (red dashed line).  

We further performed a plausibility check to constrain the effect size of starspots.  We generated flux calibrated spectra at two temperatures, and forward-modeled them to resemble the LkCa4 IGRINS and ESPaDOnS spectra in all other ways (\emph{i.e.} $\logg, \Z, \vt_{\rm ext}$), including noise and calibration parameters.  We coadded the spectra in a mixture model:  $ \mathsf{M}_{mix} = c \cdot \mathsf{M}_A(T_\textrm{eff,A}) + (1-c) \cdot \mathsf{M}_B(T_\textrm{eff,B})$.  We chose a fill factor of starspots of 30\%.  We then re-ran our single-temperature fitting procedure on the two-temperature, synthetic, noised-up data to see what stellar parameters one would na\"{\i}vely derive.  The noised-up optical data yields $\teff \sim 4100$ K, close to that of the A component.  We are awaiting the results from $K-$ band at the time of writing.


We want to measure the derived effective temperature as a function of wavelength.  The motivation for this strategy is as follows.  We assume that LkCa4 has a significant fraction of its stellar disk covered by starspots citeXX.  Such a photosphere will have an emergent spectrum composed of superpositions of relatively cool patches and warm patches.  The visible portion of the spectrum will be dominated by the warm patches, but the contrast with the starspots will be higher.  The infrared spectrum will have a lower overall spot contast, but with higher relative contribution attributable to the cool patches citeXX.  

The fit of a single-temperature forward-modeled spectrum to different regions of the spectrum could therefore conceivably yield different estimates for the effective temperature.

In principle, one could estimate the effective temperature \emph{evaluated independently for each spectral line}.  The line strength is often degenerate with other properties (\emph{e.g.} magnetic field, $\log{g}$, $[\mathrm{Fe}/\mathrm{H}]$, etc.), but \emph{on average} these correlations would be averaged over.

We measured the effective temperature in \emph{each of the 43 spectral orders} of IGRINS.  Figure \ref{fig:teffOrder} plots each derived effective temperature value and uncertainty at the center position of each order.

\begin{figure*}
	\centering
	\includegraphics[width=0.95\textwidth]{figures/teff_v_wl_example} 
	\caption{Effective temperature as derived by different orders.}
	\label{fig:teffOrder}
\end{figure*}

\subsection{Result 2: Effect size from fitting synthetic data order-by-order}

\subsection{Result 3: Direct measurement of spot properties with mixture model}

\section{Discussion}

\subsection{Interpretation of distribution of stellar properties from Results 1 + 2}

\subsection{Interpretation of star spot properties from Result 3}

\subsection{Longitudinal / latitudinal spot distributions and phase}

\subsection{Addressing limitations of assumptions in the methodology}



\appendix

\section{Testing and runing the mixture model}

Here are some specifics about what we did.

\section{Previous work}

Table REF lists measurements of LkCa4 from previous studies.
%%%%%%%%%%%%%%%%%%%%%%%%%%%%%%%%%%%%%%%%
% TABLE - History of LkCa4
%%%%%%%%%%%%%%%%%%%%%%%%%%%%%%%%%%%%%%%%
%\begin{deluxetable*}{lccccccccc}
\begin{deluxetable}{p{4cm}ccccccccc}

\tabcolsep=0.11cm
%\rotate
\tabletypesize{\footnotesize}
\tablecaption{Previous studies of LkCa4\label{tbl_history}}
\tablewidth{0pt}
\tablehead{
\colhead{Ref} &
\colhead{Band(s)} &
\colhead{Resolution} &
\colhead{Classification} &
\colhead{$T_{\rm eff}$} &
\colhead{$\log{g}$} &
\colhead{$A_V$} &
\colhead{$v\sin{i}$} &
\colhead{$v_{z}$} &
\colhead{$\log{L/L_{\odot}}$}
\\
\colhead{} &
\colhead{} &
\colhead{} &
\colhead{} &
\colhead{K} &
\colhead{} &
\colhead{} &
\colhead{km/s} &
\colhead{km/s} &
\colhead{}
}
\startdata
 \citet{herbig86} & $V$ & 100000 & K7 V & - & - & 0.68 & 26.1$\pm$2.4 & +13$\pm$4 & 0.02\\
 \citet{hartmann87} & $U$ & 100000 & T-Tauri & - & - & - & 26.1$\pm$2.4 & +16.9$\pm$2.6 & \\
 \citet{downes88} & $V$ & 13$\AA$ & Me & - & - &  - & - & - & \\
 \citet{strom89a} & IRAS & - & K7:V & - & - & 0.95 & - & - & 0.04\\
 \citet{strom89b} & $V$ & 3500 & K7:V & - & - & - & - & - & \\
 \citet{stauffer91} & $VRI$ & 500 & M1 & - & - & - & - &  - & \\
 \citet{strom94} & $V$ & - & K7 & 4000 & - & 1.25 & - & - &  0.06\\
 \citet{kenyon95} & $U$-IRAS &  & K7 & 4060 & - & 0.69 & - & - & -0.08\\
 \citet{hartigan95} & V & 25000 & K7 & 4000 & - & 0.68  & - & - & -0.07\\
 \citet{white01} & $VI$ &  & K7 & - & - & 1.21 & - & -  & -0.02 \\
 \citet{nguyen09} & $BVRI$ & 60000 & - & - & - & - & $30\pm2$ & - & \\
 \citet{nguyen12} & $BVRI$ & 60000 & - & - & - & - & $30\pm2$ & $16.0\pm4.0$ & \\
 \citet{grankin13} & $BVRI$ & & K7 & 4040 & & 0.54 & 26.1 & & -0.13 \\
 \citet{donati14} & $UBVRI$ & 68000 & & 4100$\pm50$ & 3.8$\pm$0.1 & 0.68$\pm$0.15 & 28.0$\pm$0.5 & 16.8$\pm$0.1 & -0.04$\pm$0.11 \\
 \citet{herczeg14} & $UBVRI$ & 700 & M1.3 & 3670 & & 0.35 & & & -0.29 \\
 
\enddata

\tablecomments{Some values are not original, see references to trace to original source.}
%\tablerefs{}

%\end{deluxetable*}
\end{deluxetable}

In short, here is what we know about \name.  It is a weak-lined \emph{T-Tauri} star.  It exhibits a 3.36-3.37 day period \citep{1993AJ....106.1608V,1994IBVS.4042....1G}, as evinced by photometric monitoring in $BVRI$ bands.  No periodic signal is detected at $U-$band.  Section 5 of \citet{1993AJ....106.1608V}, ``More on the nature of CTT and WTT spots'' goes into detail about the whether the photometric variability is attributable to hot or cold patches.  \name has no detected M-type-or-earlier binary companion down to $0.13''$ \citep{1993A&A...278..129L}.  \name demonstrates variability in X-rays with ROSAT \citep{1994ApJ...424..237S}, with $L_{x}=2.4$ ergs/s.


\acknowledgements
The authors thank Gregory N. Mace and Kyle Kaplan for carrying out the IGRINS observations. This research has made use of NASA's Astrophysics Data System.

{\it Facilities:} \facility{Smith (IGRINS)}

\clearpage

\bibliographystyle{apj}
\bibliography{ms}

\end{document}


